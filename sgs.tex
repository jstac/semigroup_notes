\documentclass[12pt, reqno]{amsart}

\usepackage{amsmath, amssymb, amsthm, amsfonts}

% avoid `Too many math alphabets used in version normal` error
\newcommand\hmmax{0}
\newcommand\bmmax{0}

\usepackage[mathscr]{euscript} 
\usepackage{stmaryrd} % St Mary's Road symbols font --- some extra symbols

%\usepackage{fontspec} 
\usepackage[xcharter]{newtxmath}
%\setmainfont{XCharter}

\usepackage{graphics, stackrel}
\usepackage{graphicx}

\usepackage{verbatim}
\usepackage{natbib}
\usepackage{enumitem}

%font
%\usepackage{lmodern}
%\usepackage[T1]{fontenc}
%\usepackage{fontspec} 
%\usepackage[xcharter]{newtxmath}
%\setmainfont{XCharter}
%\usepackage{mathpazo}
%\usepackage{tgpagella}

%subfloats / figures
\usepackage{caption}
\usepackage{subcaption}

% For pandas latex tables
\usepackage{booktabs}


\usepackage{fancyvrb}
\usepackage[dvipsnames,svgnames,table]{xcolor}
\usepackage{mdwlist}

\usepackage[breaklinks=true, citecolor=Brown, colorlinks=true, linkcolor=blue]{hyperref}

% lists
\usepackage{enumitem}
\setlist[enumerate]{itemsep=2pt,topsep=3pt}
\setlist[itemize]{itemsep=2pt,topsep=3pt}
\setlist[enumerate,1]{label=(\roman*)}

\usepackage{mathrsfs}  % caligraphic
%\usepackage{stix} 
\usepackage{bbm}
\usepackage{bm}        % bold symbols


%% page layout
\usepackage[left=1.25in, right=1.25in, top=1.0in, bottom=1.15in, includehead, includefoot]{geometry}

% nice inequalities
\renewcommand{\leq}{\leqslant}
\renewcommand{\geq}{\geqslant}

% inner product
\providecommand{\inner}[1]{\left\langle{#1}\right\rangle}
\providecommand{\innerp}[1]{\left\langle{#1}\right\rangle_\pi}

% lists
\usepackage{enumitem}
\setlist[enumerate]{itemsep=2pt,topsep=3pt}
\setlist[itemize]{itemsep=2pt,topsep=3pt}
\setlist[enumerate,1]{label={\upshape (\roman*)}}


\usepackage[ruled, linesnumbered]{algorithm2e}

%extra spacing
\renewcommand{\baselinestretch}{1.25}

%horizonal line
\newcommand{\HRule}{\rule{\linewidth}{0.3mm}}

% skip a line between paragraphs, no indentation
%\setlength{\parskip}{1.5ex plus0.5ex minus0.5ex}
%\setlength{\parindent}{0pt}

% footnote without a maker (blfootnote)
\newcommand\blfootnote[1]{%
  \begingroup
  \renewcommand\thefootnote{}\footnote{#1}%
  \addtocounter{footnote}{-1}%
  \endgroup
}

\DeclareMathOperator{\fix}{fix}
\DeclareMathOperator{\Span}{span}
\DeclareMathOperator{\diag}{diag}
\DeclareMathOperator*{\argmin}{arg\,min}
\DeclareMathOperator*{\argmax}{arg\,max}

\DeclareMathOperator{\cl}{cl}
\DeclareMathOperator{\Int}{int}
%\DeclareMathOperator{\overset{\circ}}{int}
\DeclareMathOperator{\Prob}{Prob}
\DeclareMathOperator{\determinant}{det}
\DeclareMathOperator{\Var}{Var}
\DeclareMathOperator{\Cov}{Cov}
\DeclareMathOperator{\graph}{graph}

% mics short cuts and symbols
\newcommand{\st}{\ensuremath{\ \mathrm{s.t.}\ }}
\newcommand{\setntn}[2]{ \{ #1 : #2 \} }
\newcommand{\fore}{\therefore \quad}
\newcommand{\preqsd}{\preceq_{sd} }
\newcommand{\toas}{\stackrel {\textrm{ \scriptsize{a.s.} }} {\to} }
\newcommand{\tod}{\stackrel { d } {\to} }
\newcommand{\too}{\stackrel { o } {\to} }
\newcommand{\toweak}{\stackrel { w } {\to} }
\newcommand{\topr}{\stackrel { p } {\to} }
\newcommand{\disteq}{\stackrel { \mathscr D } {=} }
\newcommand{\eqdist}{\stackrel {\textrm{ \scriptsize{d} }} {=} }
\newcommand{\iidsim}{\stackrel {\textrm{ {\sc iid }}} {\sim} }
\newcommand{\1}{\mathbbm 1}
\newcommand{\la}{\langle}
\newcommand{\ra}{\rangle}
\newcommand{\dee}{\,{\rm d}}
\newcommand{\og}{{\mathbbm G}}
\newcommand{\ctimes}{\! \times \!}
\newcommand{\sint}{{\textstyle\int}}

\newcommand{\given}{\, | \,}
\newcommand{\A}{\forall}

% d for integrals
\newcommand*\diff{\mathop{}\!\mathrm{d}}
\newcommand*\e{\mathrm{e}}

% nice emptyset
\let\oldemptyset\emptyset
\let\emptyset\varnothing


%\renewcommand{\times}{\! \times \!}

\newcommand{\aA}{\mathcal A}
\newcommand{\cC}{\mathscr C}
\newcommand{\sS}{\mathcal S}
\newcommand{\bB}{\mathcal B}
\newcommand{\oO}{\mathcal O}
\newcommand{\gG}{\mathcal G}
\newcommand{\hH}{\mathcal H}
\newcommand{\kK}{\mathcal K}
\newcommand{\iI}{\mathcal I}
\newcommand{\eE}{\mathcal E}
\newcommand{\fF}{\mathscr F}
\newcommand{\qQ}{\mathcal Q}
\newcommand{\tT}{\mathcal T}
\newcommand{\xX}{\mathcal X}
\newcommand{\yY}{\mathcal Y}
\newcommand{\rR}{\mathcal R}
\newcommand{\zZ}{\mathcal Z}
\newcommand{\wW}{\mathcal W}
\newcommand{\uU}{\mathcal U}
\newcommand{\lL}{\mathcal L}
\newcommand{\mM}{\mathcal M}
\newcommand{\dD}{\mathcal D}
\newcommand{\pP}{\mathcal P}
\newcommand{\vV}{\mathcal V}

\newcommand{\Bsf}{\mathsf B}
\newcommand{\Hsf}{\mathsf H}
\newcommand{\Vsf}{\mathsf V}

\newcommand{\BB}{\mathbbm B}
\newcommand{\DD}{\mathbbm D}
\newcommand{\RR}{\mathbbm R}
\newcommand{\CC}{\mathbbm C}
\newcommand{\QQ}{\mathbbm Q}
\newcommand{\NN}{\mathbbm N}
\newcommand{\GG}{\mathbbm G}
\newcommand{\UU}{\mathbbm U}
\newcommand{\TT}{\mathbbm T}
\newcommand{\YY}{\mathbbm Y}
\newcommand{\ZZ}{\mathbbm Z}
\newcommand{\HH}{\mathbbm H}
\newcommand{\MM}{\mathbbm M}
\newcommand{\PP}{\mathbbm P}
\newcommand{\VV}{\mathbbm V}
\newcommand{\EE}{\mathbbm E}


\newcommand{\bH}{\mathbf H}
\newcommand{\bT}{\mathbf T}

\newcommand{\var}{\mathbbm V}

\newcommand{\Xsf}{\mathsf X}
\newcommand{\Msf}{\mathsf M}
\newcommand{\Asf}{\mathsf A}
\newcommand{\Gsf}{\mathsf G}
\newcommand{\II}{\mathsf I}
\newcommand{\WW}{\mathsf W}

\renewcommand{\phi}{\varphi}
\renewcommand{\epsilon}{\varepsilon}

\newcommand{\bP}{\mathbf P}
\newcommand{\bQ}{\mathbf Q}
\newcommand{\bE}{\mathbf E}
\newcommand{\bM}{\mathbf M}
\newcommand{\bX}{\mathbf X}
\newcommand{\bY}{\mathbf Y}

\theoremstyle{plain}
\newtheorem{theorem}{Theorem}[section]
\newtheorem{corollary}[theorem]{Corollary}
\newtheorem{lemma}[theorem]{Lemma}
\newtheorem{proposition}[theorem]{Proposition}


\theoremstyle{definition}
\newtheorem{definition}{Definition}[section]
\newtheorem{axiom}{Axiom}[section]
\newtheorem{example}{Example}[section]
\newtheorem{remark}{Remark}[section]
\newtheorem{notation}{Notation}[section]
\newtheorem{assumption}{Assumption}[section]
\newtheorem{condition}{Condition}[section]


%\DeclareTextFontCommand{\emph}{\bfseries}

%%%%%%%%%%%%%%%%%% end my preamble %%%%%%%%%%%%%%%%%%%%%%%%%%%%%%%%%%%


\newcommand{\navy}[1]{\textcolor{blue}{#1}}




\begin{document}


\title{Semigroup Notes}

\author{JS and JY}

\date{\today}

\maketitle


In what follows, 
%
\begin{itemize}
    \item $E$ denotes a real Banach space,
    \item $\lL(E)$ is the set of bounded linear operators on $E$,
    \item $E'$ represents the dual space (i.e., the set of bounded linear
        functionals on $E$),
    \item $\| \cdot \|$ denotes either the norm on $E$ or the operator norm on
        $\lL(E)$, depending on context, and
    \item $\cC$ is the set of continuous functions from $\RR_+$ to $E$.
\end{itemize}

\section{Linear Differential Equations and Exponential Paths}

This section contains a brief review of linear ordinary differential equations
(ODEs) in Banach space when the linear operator that defines the ODE is a
bounded linear operator.

To set the stage, recall that an (autonomous) linear ODE in $\RR^n$ has the form
%
\begin{equation*}
    \dot x(t) = A x(t)
\end{equation*}
%
where $x(t)$ is an $n$-vector for each $t$ and $A$ is $n \times n$.  The matrix
$A$ is sometimes called the ``vector field.''  A first step to extend these
concepts to an abstract Banach space is to set $\dot x(t) = A x(t)$, where
$x(t)$ is understood as a point in the space and $A$ is a \emph{bounded} linear
operator from the space to itself.  (This includes the finite-dimensional case,
including the one-dimensional case as taught in elementary lectures on ODEs,
since all linear operators on finite-dimensional spaces are bounded.  The
derivative $\dot x$ is defined below.)

It turns out that the boundedness restriction leaves out many interesting and
important systems.  On the other, it illustrates the ideal case, and
highlights some of the properties we hope to discover for the solutions of
general ODEs (driven by possibly unbounded linear operators) in Banach space.  




\subsection{Preliminaries}\label{ss:prel}

First we recall some facts concerning calculus for Banach space valued
functions. To this end, let $u$ be an element of $\cC$. If, for some $t > 0$,
the limit
%
\begin{equation*}
    \dot u(t)
    \coloneq \frac{\diff }{\diff t} u(t)
    \coloneq \lim_{h \to 0} \frac{\| u(t + h) - u(t) \|}{h}
\end{equation*}
%
exists, then we say that $u$ is \navy{differentiable} at $t$ and call $\dot u(t)$
the \navy{derivative} of $u$ at $t$.  In the case where $t=0$ we use the same
definitions after replacing $h \to 0$ with $h \downarrow 0$.
We say that $u$ is \navy{differentiable} on $\RR_+$ if $u$ is differentiable at
$t$ for all $t \in \RR_+$.

Next we define an integral for continuous Banach space valued functions. Fix $u
\in \cC$.  the (Riemann) \navy{integral} of $u$ over $[a,b]$ is the unique point
$I \in E$ such that
%
\begin{equation}\label{eq:rieint}
    \int_a^b \inner{u(s), u'} \diff s = \inner{I, u'} 
    \quad \text{ for all } u' \in E'
\end{equation}
%
In \eqref{eq:rieint}, the left hand side is an ordinary Riemann integral for
real-valued functions.  
We write $I = \int_a^b u(s) \diff s$, so that $
\int_a^b \inner{u(s), u'} \diff s = \inner{\int_a^b u(s) \diff s, u'}$ for all
$u' \in E$.  A proof of existence and uniqueness, as well as the properties
listed below, can be found in Section~5.1.2 of \cite{buhler2018functional}.

\begin{lemma}\label{l:riemann}
    If $u$ is continuous on $\RR_+$ and $a, b \in \RR_+$ with $a < b$, then
    %
    \begin{enumerate}
        \item $\int_a^c u(s) \diff s 
            = \int_a^b u(s) \diff s + \int_b^c u(s) \diff s$ whenever $a < b < c$,
        \item $\int_a^b u(s + c) \diff s = \int_{a+c}^{b+c} u(s)
            \diff s$ whenever $c \geq 0$,
        \item $\| \int_a^b u(s) \diff s \| \leq \int_a^b \|
            u(s)  \| \diff s$,
        \item $u(t) = \lim_{h \to 0} \, (1/h) \int_t^{t+h} u(s) \diff s$ for all
            $t \in \RR_+$, and
        \item $L \in \lL(E)$ implies $L \int_a^b u(s) \diff s = \int_a^b Lu(s) \diff s $.
    \end{enumerate}
    %
    If, in addition, $u$ is continuously differentiable on $[a, b]$ with
    derivative $\dot u$, then
    %
    \begin{equation}\label{eq:ftcb}
        u(b) = u(a) + \int_a^b \dot u(s) \diff s.
    \end{equation}
    %
\end{lemma}



\subsection{Exponentials of Linear Operators}

The scalar exponential of $a \in \RR$ can be defined by $\exp(a)
\coloneq \sum_{n=0}^\infty \frac{a^n}{n!}$.  Analogously, the \navy{exponential}
of $A \in \lL(E)$ is given by
%
\begin{equation}\label{eq:expfun} 
    \exp(A) 
    \coloneq \sum_{n=0}^\infty \frac{A^n}{n!}
    = I + A + \frac{A^2}{2!} + \cdots
\end{equation}
%
We list some properties of the exponential below.

\begin{lemma}\label{l:expcom}
    Let $A$ and $B$ be elements of $\lL(E)$.  The following properties hold:
    %
    \begin{enumerate}
        \item $\exp(A)$ is a well-defined element of $\lL(E)$  and $\| \exp(A) \| \leq \exp(\|A\|)$.
        \item If $AB = BA$, then $\exp(A + B) =
            \exp(A) \exp(B)$.
        \item $\exp(0) = I$.
    \end{enumerate}
    %
\end{lemma}

It is immediate from (ii) that if $m$ is any positive integer, then $\exp(mA) =
(\exp(A))^m$.

The next lemma considers exponential flows in $\lL(E)$.

\begin{lemma}\label{l:expcom2}
    Fix $A \in \lL(E)$ and consider the function $e(t) \coloneq \exp(tA)$ from
    $\RR$ to $\lL(E)$.  The following statements are true:
    %
    \begin{enumerate}
        \item The map $e$ is differentiable in $t$, with
            %
            \begin{equation}\label{eq:matdiffl}
                \dot e(t) = A e(t) = e(t) A.
            \end{equation}
            %
        \item The fundamental theorem of calculus holds, in the sense that
            %
            \begin{equation}\label{eq:ftcexp}
                e(b) 
                 = e(a) + \int_a^b \dot e(s) \diff s \quad \text{for all } a \leq b.
            \end{equation}
            %
    \end{enumerate}
    %
\end{lemma}



For example, to prove (i) of Lemma~\ref{l:expcom2}, we fix $A \in \lL(E)$ and consider $e(t) = \exp(t
A)$. Applying (ii) of Lemma~\ref{l:expcom} yields
%
\begin{equation*}
    \frac{e(t + h) - e(t)}{h}
    = \frac{\exp(tA + hA) - \exp(tA)}{h}
    = \exp(tA) \frac{\exp(hA) - I}{h}
\end{equation*}
%
Using the definition \eqref{eq:expfun} and taking the limit as $h \downarrow
0$ gives $\dot e(t) = e(t) A $.  A small variation on the argument 
shows that $\dot e(t) = A e(t)$ also holds.


\subsection{Initial Value Problems}\label{ss:boundedivp}

Now fix $A \in \lL(E)$ and consider the \navy{initial value problem} (IVP) $\dot
u(t) = A u(t)$ for all $t \geq 0$ with given initial condition $u(0) = u_0 \in
E$.  A solution to this problem is $u \in \cC$ that is differentiable and 
obeys the IVP conditions.

\begin{proposition}\label{p:expsol}
    Given $A \in \lL(E)$, the function $u$ defined by $u(t) = \exp(tA) u_0$ is
    the unique solution to the IVP stated above.
\end{proposition}

That $u$ solves the IVP is immediate from the properties in
Lemmas~\ref{l:expcom}--\ref{l:expcom2}.  
We prove uniqueness in a more general setting below.

Proposition~\ref{p:expsol} gives a complete description of the solution of ODEs
in Banach space when the vector field $A$ is a bounded linear operator. As
well as studying flows of the form $t \mapsto \exp(tA) u_0$ in $E$ we can also
directly analyze the dynamics of flows $t \mapsto \exp(tA)$ in $\lL(E)$. We
notice that these flows are continuous (in fact differentiable, by
Lemma~\ref{l:expcom2}) and obey $\exp((s + t)A) = \exp(tA) \exp(sA)$ for all $s,
t \geq 0$.  The latter property is called the \emph{semigroup} property of the
flow.  

In order to generalize these ideas and handle ODEs with unbounded vector fields, 
we now study a more general notion of continuous semigroups in $\lL(E)$.


\section{Introduction to Semigroups}

Add roadmap.

\subsection{Definitions}

Let $(S_t) \coloneq (S_t)_{t \geq 0}$ be a family of
linear operators in $\lL(E)$ with index $t \in \RR_+$. $(S_t)$ is called 
an \navy{algebraic operator (AO) semigroup} if 
$S_0$ is the idendity and $(S_t)$ has the semigroup property
%
\begin{equation*}
    S_{s + t} = S_t \circ S_s
    \quad \text{for all } s, t \in \RR_+.
\end{equation*}
%
If, in addition, $t \mapsto S_t u$ is continuous for all $u \in E$, then $(S_t)$
is called a \navy{$C_0$-semigroup}.  When $E$ is understood, we say that $(S_t)$
is a $C_0$-semigroup. Given $u \in E$, the function $t \mapsto S_t u$ is
called the \navy{orbit} or \navy{trajectory} of $u$ under $(S_t)$.
The point $u$ is called the \navy{initial condition}.



\subsection{Continuity Results}


Let $K$ be a compact subset of $\RR$ and let $\{S_t\}_{t \in K}$ be a subset of
$\lL(E)$.  The following result is from \cite{engel2006short}.

\begin{lemma}\label{l:contin}
    The following statements are equivalent:
    %
    \begin{enumerate}
        \item The map $t \mapsto S_t u$ is continuous on $K$ for all $u \in E$.
        \item $\|S_t\|$ is bounded over $t \in K$ and there exists a dense subset
            $D$ of $E$ such that $t \mapsto S_t u$ is continuous on $K$ for all $u \in D$.
        \item For any compact $C \subset E$, the map $(t, u) \mapsto S_t u$ is
            uniformly continuous on $K \times C$.
    \end{enumerate}
\end{lemma}

\begin{proof}
    ((i) $\implies$ (ii)) By (i), for any $u \in E$, the map $t \mapsto S_t u$ is
    continuous on a compact set and, therefore, its image is bounded in $E$.  
    Hence, by the uniform boundedness principle, $\|S_t\|$ is bounded over $t
    \in K$.  The statement in (ii) regarding continuity is obvious.

    ((ii) $\implies$ (iii)).  Fix compact $C \subset E$ and $\epsilon > 0$.
    We metrize $K \times C$ by setting $d((s, u), (t, v)) = \| u - v\| \vee |s-t|$.
    Choose $M \in \NN$ such that $\|S_t\| \leq M$ for all $t \in K$.  Let $D$ be the
    dense set in (ii) and observe that the set of open balls $B(u, \epsilon/M)$
    over $u \in D$ provides an open cover of $C$.  As such, we can choose a finite
    set $D_F \subset D$ such that $C$ is contained in $\cup_{u \in D_F} B(u,
    \epsilon/M)$. Since, for each $u \in D_F$, the map
    $t \mapsto S_t u$ is continuous on a compact set, it is also uniformly
    continuous.  As a result, given $u \in D_F$, we can select a $\delta_u > 0$ such that
    %
    \begin{equation*}
        |s - t| < \delta_u \implies \| S_s u - S_t u \| < \epsilon.
    \end{equation*}
    %
    Let $\delta$ be the minimum of $\{\delta_u\}_{u \in D_F}$ and $\epsilon / M$.
    If we take $u, v \in C$ and $s, t \in K$ with $d((s, u), (t, v)) < \delta$, then,
    choosing $w \in D_F$ with $\|u - w \| < \epsilon / M$, we have
    %
    \begin{align*}
        \| S_s u - S_t v \|
        & \leq \| S_s u - S_s w \| 
            + \| S_s w - S_t w \| 
                + \| S_t w - S_t v \|
                \\
        & = \| S_s (u - w) \| 
            + \| S_s w - S_t w \| 
                + \| S_t (w - v) \|
                \\
        & < M (\epsilon / M) + \epsilon + M (2 \epsilon / M) = 4 \epsilon.
    \end{align*}
    %
    Hence $(t, u) \mapsto S_t u$ is
            uniformly continuous on $K \times C$, as claimed.

    ((iii) $\implies$ (i)) This implication is trivial (take $C$ to be a
    singleton).
\end{proof}

\begin{lemma}\label{l:ubfc}
    If $(S_t)_{t \geq 0}$ is a $C_0$-semigroup on $E$, then
        $\sup_{t \leq \delta} \| S_t \| < \infty$ for all $\delta > 0$.
\end{lemma}

\begin{proof}
    We first claim there exists an $\epsilon > 0$ such that $
    \sup_{t \leq \epsilon} \| S_t \| < \infty$.  Indeed, if no such $\epsilon$
    exists, then there exists a sequence $t_n \to 0$ such that 
    $\|S_{t_n}\|$ is unbounded.  But then, by the principle of uniform
    boundedness, there exists a $u \in E$ such that $\|S_{t_n} u\|$ is
    unbounded.  This contradicts the continuity property of $C_0$-semigroups.

    Now let $\epsilon$ be as above and choose $M \in \NN$ with $\| S_t \| \leq
    M$ whenever $t \leq \epsilon$.  Fix $k \in \NN$ and $t \leq k \epsilon$.
    Since $S_t$ is $k$ compositions of $S_{t/k}$, and since $t/k < \epsilon$, 
    the semigroup property yields $\| S_t \| \leq k M$.  Hence $t \mapsto S_t$
    is bounded on $[0, k \epsilon]$.  Since $k$ was an arbitrary element of
    $\NN$, this proves the claim in Lemma~\ref{l:ubfc}.
\end{proof}

\begin{lemma}\label{l:semiequivcon}
    If $(S_t)$ is an AO semigroup on $E$, then the following
    statements are equivalent:
    %
    \begin{enumerate}
        \item $(S_t)$ is a $C_0$-semigroup on $E$.
        \item $\lim_{t \downarrow 0} S_t u = u$ for all $u \in E$.
    \end{enumerate}
\end{lemma}

\begin{proof}
    That (i) implies (ii) is obvious.  For the reverse implication, fix $u \in E$ and $t > 0$.  We
    need to show that $\|S_{t+h} u - S_t u\| \to 0$ as $h \to 0$.  Suppose first
    that $h \downarrow 0$.  Then
    %
    \begin{equation*}
        \|S_{t+h} u - S_t u\|  
        = \|S_t S_h u - S_t u\|  
        \leq \|S_t \| \| S_h u - u\|  \to 0.
    \end{equation*}
    %
    If, on the other hand $h \uparrow 0$, then
    %
    \begin{equation*}
        \|S_{t+h} u - S_t u\|  
        = \|S_{t+h} u - S_{t + h} S_{-h} u\|  
        \leq \|S_{t+h} \| \| u - S_{-h} u\| \to 0 .
    \end{equation*}
    %
    In the last step we used the fact that $\|S_{t+h} \| $ is bounded over $h$
    by Lemma~\ref{l:ubfc}.
\end{proof}

\begin{lemma}\label{l:aose}
    Let $(S_t)_{t \geq 0}$ be an AO semigroup on $E$.
    If there exists a dense subset $D$ of $E$ such that $\lim_{t \downarrow 0}
    S_t u = u$ for all $u \in D$ and, in addition, $\sup_{t \leq \delta} \|S_t
    \| < \infty$ for some $\delta > 0$, then $(S_t)_{t \geq 0}$ is a
    $C_0$-semigroup.
\end{lemma}

\begin{proof}
    Fix $u \in E$.  By Lemma~\ref{l:semiequivcon} it suffices to show
    that, for a given sequence $t_n \downarrow 0$, we have $S_{t_n} u \to u$ as
    $n \to 0$. 
    To see that this holds, fix $t_n \downarrow 0$ and choose a compact subset
    $K$ of $\RR_+$ such
    that $\{t_n\} \subset K$.  Since $K$ is compact, $K \ni t \mapsto S_t w$ is
    continuous when $w \in D$, and $\|S_t \|$ is bounded over $t \in K$,
    Lemma~\ref{l:contin} implies that $K \ni t \mapsto S_t u$ is continuous.
    In particular, $S_{t_n} u \to u$ as $n \to 0$. 
\end{proof}



\subsection{Examples}


\subsubsection{Left-Shift Semigroups}

Let $C_0(\RR_+)$ be the set of all continuous real-valued functions $f$ on
$\RR_+$ with $f(x) \to 0$ as $x \to \infty$.  The set $C_0(\RR_+)$ is paired
with the supremum norm.  Consider the \navy{left
translation semigroup} given by $(S_t f)(x) = f(x + t)$.

\begin{lemma}\label{l:ltsemi}
    $(S_t)$ is a $C_0$-semigroup on $C_0(\RR_+)$.  
\end{lemma}

\begin{proof}
    Evidently $S_0 f = f$.  The semigroup property holds because, for $s, t
    \geq 0$, we have
    %
    \begin{equation*}
        (S_{s + t} f)(x) = f(x + s + t) = (S_t (S_s f))(x).
    \end{equation*}
    %
    Regarding continuity, fix $f \in C_0(\RR_+)$ and let $(t_n)$ be a real sequence
    with $t_n \downarrow 0$.  Fix $\epsilon > 0$.  Since $f$ is uniformly
    continuous, we can select a $\delta > 0$ such that $|f(x) - f(y)| < \epsilon$
    whenever $|x-y|<\delta$.  Let $N \in \NN$ be such that $t_n < \delta$ when
    $n \geq N$.  Then, for $n \geq N$,
    %
    \begin{equation*}
        \| S_{t_n} f - f \|
        = \sup_x | f(x + t_n)  - f(x) |
        < \epsilon.
    \end{equation*}
    %
    Hence $S_t f \downarrow f$ and $(S_t)$ is a $C_0$-semigroup.
\end{proof}


Let $C_0^1(\RR_+)$ be the set of all continuously differentiable $f \in
C_0(\RR+)$ with $f' \in C_0(\RR+)$.  The set $C_0^1(\RR_+)$ is paired
with the norm $\|f\| = \sup_x |f(x)| + \sup_x |f'(x)|$.  

\begin{lemma}\label{l:ltsemi2}
    $(S_t)$ is a $C_0$-semigroup on $C_0^1(\RR_+)$.  
\end{lemma}

\begin{proof}
    In view of Lemma~\ref{l:ltsemi}, we only need to check continuity.
    Fixing $f \in C_0^1(\RR_+)$, we have
    %
    \begin{equation*}
        \| S_t f - f \|
        = \sup_x | f(x + t)  - f(x) | + \sup_x | f'(x + t)  - f'(x) |
    \end{equation*}
    %
    Since $f$ and $f'$ are both in $C_0(\RR_+)$, the proof of
    Lemma~\ref{l:ltsemi} implies that both terms on the right hand side
    converge to zero as $t \downarrow 0$.  Hence continuity holds.
\end{proof}

\subsubsection{Right-Shift Semigroups}\label{ss:rssemi}

Here we discuss right-shift semigroups.  We will embed them in a space of
integrable functions.  Below $\lambda$ denotes Lebesgue measure.

Let $C_c(\RR)$ be the set of all continuous real-valued functions $f$ on
$\RR$ that vanish off a compact set.  Let $L_1(\RR)$ be the set of 
Borel measurable real-valued functions on $\RR$ with $\| f \| \coloneq \int
|f| \diff \lambda < \infty$.  Let $S_t$ be the linear operator on $L_1(\RR)$
defined by 
%
\begin{equation*}
    (S_t f)(x) = f(x - t)
    \qquad (x \in \RR, \; t \geq 0).
\end{equation*}
%

\begin{lemma}\label{l:rtsemi}
    $(S_t)$ is a $C_0$-semigroup on $L_1(\RR)$.  
\end{lemma}

\begin{proof}
    It is simple to confirm that $(S_t)$ is an AO semigroup on $L_1(\RR)$.
    Regarding continuity, note that $\|S_t f\| = \int |f(x-t)| \diff x = \| f\|$, so
    $\|S_t\|$ is bounded in $t$.  Since $C_c(\RR)$ is dense in $L_1(\RR)$ under this
    norm, Lemma~\ref{l:aose} implies that, to show $(S_t)$ is a $C_0$-semigroup
    on $L_1(\RR)$, it suffices to show that $\| S_t f - f \| \to 0$ for any $f \in
    C_c(\RR)$.

    To this end, fix $f \in C_c(\RR)$ and let $K$ be a compact set such that $f$
    vanishes off $K$.  Fix $\epsilon > 0$.  By uniform continuity, we can take a
    $\delta > 0$ such that $|x-y| < \delta$ implies $|f(x) - f(y)| < \epsilon /
    \lambda(K)$.  If $t < \delta$, then
    %
    \begin{equation*}
        \| S_t f - f \| 
        = \int | f(x-t) - f(x) | \diff x
        \leq \lambda(K) \frac{\epsilon}{\lambda(K)} = \epsilon.
    \end{equation*}
    %
    This completes the proof of $C_0$-continuity of $(S_t)$ on $L_1(\RR)$.
\end{proof}


\subsubsection{Multiplication Semigroups}\label{ss:multisemi}

Let $(\Xsf, \bB, \mu)$ be a $\sigma$-finite measure space and let
$\phi$ be a measurable map from $\Xsf$ to $\RR_+$.  Define
%
\begin{equation*}
    S_t f (x)= \exp(-t \phi(x)) f(x) 
    \qquad (x \in \Xsf, \; t \geq 0).
\end{equation*}
%
The family $(S_t)$ is called a \navy{multiplication semigroup}.

\begin{lemma}\label{l:multisemi}
    $(S_t)$ is a $C_0$-semigroup on $L_1(\Xsf, \bB, \mu)$.
\end{lemma}

\begin{proof}
    It is simple to confirm that $(S_t)$ is an AO semigroup on $L_1(\RR)$.
    Regarding continuity, fix $f \in L_1(\Xsf, \bB, \mu)$ and observe that
    %
    \begin{equation*}
        \| S_t f - f \|
        = \int |f(x)| | \exp(-t\phi(x)) - 1 | \mu(\diff x).
    \end{equation*}
    %
    It follows from the dominated convergence theorem that this integral
    converges to zero as $t \downarrow 0$.  Hence $(S_t)$ is a $C_0$-semigroup
    on $L_1(\Xsf, \bB, \mu)$.
\end{proof}



\subsubsection{Uniformly Continuous Semigroups}\label{ss:ucsemi}

Fixing $A \in \lL(E)$, consider the family of linear operators on $E$ given
by 
%
\begin{equation*}
    S_t u = \exp(t A) u
    \qquad (u \in E, \; t \geq 0)
\end{equation*}
%
We recall that the exponential function $\phi(t) \coloneq \exp(t A)$ 
%
\begin{enumerate}
    \item obeys $\phi(0) = I$ and $\phi(s + t) = \phi(t) \phi(s)$ for all $s, t
        \in \RR$; and
    \item is continuous as a map from $\RR$ to $\lL(E)$.
\end{enumerate}
%
From (i) we can easily confirm that $(S_t)$ is an algebraic operator semigroup
on $E$.  Regarding continuity, (i) and (ii) imply that
%
\begin{equation}\label{eq:ucsemi}
    \lim_{t \downarrow 0} \| S_t - I \| = 0.
\end{equation}
%
It follows from \eqref{eq:ucsemi} that $(S_t)$ is a $C_0$-semigroup on $E$.

Any operator semigroup $(S_t)$ on $E$ obeying \eqref{eq:ucsemi} is called a
\navy{uniformly continuous semigroup}.  In fact no other examples exist:

\begin{theorem}\label{t:ucsemi}
    If $(S_t)$ is a uniformly continuous semigroup on $E$, then there exists an
    $A \in \lL(E)$ such that $S_t u = \exp(t A) u$ for all $u \in E$ and $t \geq 0$.
\end{theorem}

The proof of Theorem~\ref{t:ucsemi} can be found 2(b) of \cite{engel2006short}.


\section{Generators and Resolvents}

Add roadmap.


\subsection{Infinitesmial Generators}

Let $E$ be a Banach space and let $(S_t)$ be an AO semigroup on $E$.  Set
%
\begin{equation*}
    \dD(A) =
    \left\{
        u \in E \st 
        \lim_{t \downarrow 0} \frac{S_t u - u}{t} \text{ exists}
    \right\}
    \quad \text{and} \quad
       Au = \lim_{t \downarrow 0} \frac{S_t u - u}{t} \text{ on }
       \dD(A).
\end{equation*}

The map $A$ is called the \navy{infinitesimal generator} of $(S_t)$ and $\dD(A)$
is its domain.  

\begin{lemma}\label{l:dals}
    The set $\dD(A)$ is a linear subspace of $E$ and $A$ is linear on $\dD(A)$.
\end{lemma}

\begin{proof}
    Fix $\alpha, \beta \in \RR$ and $u,v \in \dD(A)$.  Let $w = \alpha u + \beta v$.
    Since
    %
    \begin{equation*}
        \frac{S_t w - w}{t} 
        = \alpha \frac{S_t  u - u}{t} + \beta \frac{S_t  v - v}{t},
    \end{equation*}
    %
    we see that $w \in \dD(A)$ and, moreover, $Aw = \alpha Au + \beta Av$.
\end{proof}

The next result shows that any trajectory of $(S_t)$ starting from a point $\bar
u$ in $\dD(A)$ obeys a linear differential equation evolving in $\dD(A)$, with $A$
as the ``vector field.''


\begin{lemma}\label{l:diffpath}
    If $(S_t)$ is a $C_0$-semigroup and $u \in E$,  then, for all $t \geq 0$,
    %
    \begin{equation}\label{eq:pathfrome}
        \int_0^t S_s \, u \diff s \in \dD(A) \quad \text{and} \quad
        S_t \, u = u + A \int_0^t S_s \, u \diff s.
    \end{equation}
    %
\end{lemma}


\begin{proof}
    We begin by proving \eqref{eq:pathfrome}. Fix $u \in E$ and $t \geq 0$.
    For $h > 0$, using properties of the integral described in
    Lemma~\ref{l:riemann}, we have
    %
    \begin{align*}
            S_h \int_0^t S_s \, u \diff s - \int_0^t S_s \, u \diff s
        & = 
            \int_0^t S_{h + s} \, u \diff s - \int_0^t S_s \, u \diff s
        \\
        & = 
            \int_h^{t+h} S_s \, u \diff s - \int_0^t S_s \, u \diff s
        = 
            \int_t^{t+h} S_s \, u \diff s - \int_0^h S_s \, u \diff s
    \end{align*}
    %
    Multiplying by $(1/h)$ and taking the limit as $h
    \downarrow 0$ gives $A \int_0^t S_s \, u \diff s = S_t u - u$. Thus
    \eqref{eq:pathfrome} is verified.
\end{proof}


\begin{proposition}\label{p:diffpath2}
    If $(S_t)$ is a $C_0$-semigroup, then $\dD(A)$ is dense in $E$.  Moreover,  for any
    $u \in \dD(A)$ and $t \geq 0$,
    %
    \begin{enumerate}
        \item $S_t u \in \dD(A)$,
        \item the map $s \mapsto S_s u$ is differentiable on $\RR_+$ and
            %
            \begin{equation}\label{eq:eqdiff}
                \frac{\diff}{\diff t} S_t u
                = A S_t u = S_t A u,
                \quad \text{and}
            \end{equation}
            %
        \item $S_t u = u + \int_0^t S_s A u \diff s$ for all $u \in \dD(A)$ and $t \geq 0$.
    \end{enumerate}
    %
\end{proposition}

\begin{proof}
    To see that $\dD(A)$ is dense, consider $v_h \coloneq (1/h) \int_0^h S_s \,
    u \diff s$ with $h \geq 0$. By Lemma~\ref{l:riemann},  $v_h \to u$ as $h \downarrow 0$.
    Moreover, \eqref{eq:pathfrome} implies that $v_h$ is in $\dD(A)$ for
    all $h > 0$.   Hence $\dD(A)$ is dense in $E$.


    Regarding (i), fix $t \geq 0$ and $u \in \dD(A)$.  We have
    %
    \begin{equation*}
        A S_t u
        = \lim_{h \downarrow 0} \frac{S_h S_t u - S_t u}{h}
        = \lim_{h \downarrow 0} \frac{S_t S_h u - S_t u}{h}
        = \lim_{h \downarrow 0} S_t \frac{S_h u - u}{h}
        =  S_t A u,
    \end{equation*}
    %
    where the last equality is by continuity of $S_t$ (i.e., $S_t \in \lL(E)$).
    This proves both (i) and the second equality in \eqref{eq:eqdiff}.

    We now show that $s \mapsto S_s u$ is differentiable at $t \in \RR_+$ with
    derivative $S_t A u$, which will complete the proof of (ii). We choose $t >
    0$  because the case $t=0$ is trivial. The right derivative of $S_t u$ is 
    %
    \begin{equation*}
        \lim_{h \downarrow 0} \frac{S_{t + h} u - S_t u}{h}
        = \lim_{h \downarrow 0} \frac{S_t S_h u - S_t u}{h}
    \end{equation*}
    %
    which has already been shown to equal $S_t A u$.

    Regarding the left-hand derivative, we take $0 < h < t$ and use
    %
    \begin{equation*}
        \frac{S_{t-h} u - S_t u}{-h} - S_t A u
        = S_{t-h} \left( \frac{u - S_h u}{-h} - S_h A u \right)
    \end{equation*}
    %
    and the boundedness of $\| S_t \|$ over bounded sets (Lemma~\ref{l:ubfc}) to
    obtain a finite $M$ such that
    %
    \begin{equation*}
        \left\|
            \frac{S_{t-h} u - S_t u}{-h} - S_t A u
        \right\|
        \leq M
        \left\|
            \frac{u - S_h u}{-h} - A u
        \right\|
        +
        M
        \left\|
            A u - S_h A u
        \right\|.
    \end{equation*}
    %
    Both terms on the right converge to zero in $h$, so we have proved that
    the left derivative is also to $S_t A u$.  This completes the proof of (ii).

    Regarding (iii), let $\phi(s) = S_s u$ for all $s \geq 0$.  We have shown in
    (ii) that $\phi$ is differentiable on $\RR_+$ with $\dot \phi(s) = S_s A u$.  
    With this notation, the claim in (iii) can be expressed as
    $\phi(t) = \phi(0) + \int \dot \phi(s)
    \diff s$. Since $s \mapsto S_s A u$ is continuous in $s$, $\phi$ is also continuously
    differentiable.  The claim in (iii) now follows from \eqref{eq:ftcb}.
\end{proof}

The next result shows that semigroups are uniquely identified by their
generators.

\begin{proposition}\label{p:sgufg}
    Let $(S_t)$ and $(T_t)$ be two $C_0$-semigroups on $E$ with infinitesimal
    generators $(A, \dD(A))$ and $(B, \dD(B))$.  If $(A, \dD(A)) = (B, \dD(B))$,
    then $(S_t) = (T_t)$.
\end{proposition}

\begin{proof}
    Let the statement hold, so that $(S_t)$ and $(T_t)$ are $C_0$-semigroups
    $(A, \dD(A))$. Fix $t > 0$.  We claim that $S_t = T_t$ on $E$. Since
    $\dD(A)$ is dense in $E$, it suffices to show that $S_t$ and $T_t$ agree on
    $\dD(A)$.

    To this end, fix $u \in \dD(A)$ and let $s \mapsto v(s)$ be a differentiable
    function on $\RR_+$ with $\dot v = A v$ and $v(0) = u$.  Define $w(s)
    \coloneq S_{t-s} v(s)$ for all $s \in (0, t)$.
    For $h$ close to zero we have 
    %
    \begin{align*}
        w(s + h) - w(s)
        & = S_{t-s-h} v(s + h) - S_{t-s} v(s)
        \\
        & = S_{t-s-h} (v(s + h) - v(s)) + S_{t-s-h} v(s) - S_{t-s} v(s)
    \end{align*}
    %
    Dividing by $h$ gives
    %
    \begin{equation}\label{eq:wsh}
        \frac{w(s + h) - w(s)}{h}
        = S_{t-s-h} \frac{v(s + h) - v(s)}{h} 
            - S_{t-s} \frac{S_{-h} v(s) - v(s)}{-h}
    \end{equation}
    %
    Consider the first term in \eqref{eq:wsh}.  Letting
    %
    \begin{equation*}
        d(h) \coloneq
        \left\|
           S_{t-s-h} \frac{v(s + h) - v(s)}{h} 
           - S_{t-s} A v(s)
        \right\|
    \end{equation*}
    %
    and, choosing $M$ suitably large, we have
    %
    \begin{align*}
        d(h) 
        & \leq 
        \|S_{t-s-h}\|
        \left\|
            \frac{v(s + h) - v(s)}{h} 
           - S_h A v(s)
        \right\|
        \\
        & \leq M
        \left\|
            \frac{v(s + h) - v(s)}{h} 
           - A v(s)
        \right\|
        + M \| Av(s) - S_h A v(s) \|
    \end{align*}
    %
    This confirms that $d(h) \to 0$ as $h \to 0$, which means that the first
    term in \eqref{eq:wsh} converges to $S_{t-s} A v(s)$.

    The second term also converges to
    $S_{t-s} A v(s)$. Hence $w$ is differentiable on $(0, t)$ with $\dot w = 0$.
    It follows from \eqref{eq:ftcb} that $w(s) = w(0) = S_t u$ for all $s < t$.
    Hence $S_{t-s} v(s) = S_t u$ for all $s < t$.  Taking $s \to t$ and applying
    continuity gives $v(t) = S_t u$.  

    To complete the proof we consider the case $v(s) = T_s u$.  According to
    Proposition~\ref{p:diffpath2}, the function $v$ is differentiable on $\RR_+$
    with $\dot v = A v$ and $v(0) = v$. Hence $T_t u = S_t u$.
    This confirms that $S_t$ and $T_t$ agree on $\dD(A)$.
\end{proof}



Let's now translate the results above in to findings for IVPs with potentially
unbounded generators.  We confirm that many of the results from the bounded
case (see \S\ref{ss:boundedivp}) either carry over or have direct analogs with
the general case.

In the next theorem, $C_1(\RR_+, E)$ is the set of continuously differentiable
functions from $\RR_+$ to $E$.

\begin{theorem}\label{t:ivps}
    Let $(S_t)$ be a $C_0$-semigroup with infinitesimal generator $(A, \dD(A)$
    and let $u_0$ be a point in $\dD(A)$.  The unique solution in $C(\RR_+, E)$
    of the initial value problem $\dot u = A u$ with $u(0) = u_0$ is the
    function defined by $u(t) \coloneq S_t u_0$. Moreover, $u(t)$ is continuously
    differentiable on $\RR_+$ and 
    %
    \begin{equation*}
       \dot u(t) = A S_t u = S_t A u_0.
    \end{equation*}
    %
\end{theorem}

\begin{proof}
    Most of the claims in Theorem~\ref{t:ivps} follow directly from
    Proposition~\ref{p:diffpath2}.  The only missing components is uniqueness.
    To prove it, fix $u_0 \in \dD(A)$ and let $v$ be a differentiable function from
    $\RR_+$ to $E$ with $\dot v(t) = A v (t)$ for all $t$ and $v(0) = u_0$.
    We already showed in the proof of Proposition~\ref{p:sgufg} that for any
    such function we have $v(t) = S_t u_0 = u(t)$.  Hence uniqueness also holds.
\end{proof}



\subsection{Resolvents} 
Let $A$ be the infinitesmial generator of a $C_0$-semigroup $(S_t)$. Since $A$ is closed linear operator, we define the \navy{Resolvent} of $(S_t)$ by
$$R_{\lambda}(A)=(\lambda I-A)^{-1}$$
where $R_{\lambda}(A): E\rightarrow\dD(A)$ and $\lambda\in\rho(A)$. Thus, $R_{\lambda}(A) \in\mathcal{L}(E)$.

\begin{proposition} 
	Let $A$ be the infinitesmial generator of a $C_0$-semigroup $(S_t)$ satisfying $\| S_t\|\leq M e^{at}$ for all $t\geq 0$. The following hold:
	\begin{enumerate}
		\item $\{\lambda\in \mathbb{C}:\Re(\lambda)>a\}\subset \rho(A)$
		\item for all $\Re(\lambda)>a$
		$$R_{\lambda}(A)=\int_{0}^{\infty} e^{-\lambda t}S_t \diff t$$
		
		\item for all $\Re(\lambda)>a$
		$$\|R_{\lambda}(A)\|\leq\frac{M}{\Re(\lambda)-a}$$
	\end{enumerate}
\end{proposition}
\begin{proof}
	 Let $T_{\lambda}=\int_{0}^{\infty} e^{-\lambda t}S_t \diff t$ and we need to show $T_{\lambda}=(\lambda I-A)^{-1}$ for  all $\Re(\lambda)>a$. That is, for all $u\in E$
	 \begin{equation*}
	 	T_{\lambda}(\lambda I-A)u=u
	 	\quad \text{and} \quad
	 	(\lambda I-A)T_{\lambda}u=u
	 \end{equation*}
 which is equivalent to 
    \begin{equation}\label{eq:rvt}
   	T_{\lambda}Au=AT_{\lambda}u=\lambda T_{\lambda}u-u
     \end{equation}
 We first show $T_{\lambda}$ is bounded.
     \begin{align*}
 	\|T_{\lambda}u\|
 	&=\|\int_{0}^{\infty} e^{-\lambda t}S_tu \diff t\|\leq \int_{0}^{\infty} \|e^{-\lambda t} \|\cdot\|S_tu\| \diff t\\
 	&\leq  	\|u\|M\int_{0}^{\infty} \e^{(a-\Re(\lambda)) t} \diff   t=\frac{M}{\Re(\lambda)-a} \|u\| 
    \end{align*}
Since $T_{\lambda}$ and $(S_t)$ are bounded,
    \begin{align*}
    	T_{\lambda} Au
    	&=\int_{0}^{\infty} e^{-\lambda t}S_t \lim_{h \downarrow 0}\frac{S_h-I}{h}u\diff t\\
    	&=\lim_{h \downarrow 0}\int_{0}^{\infty} e^{-\lambda t}S_t \frac{S_h-I}{h}u\diff t\\
    	&=\lim_{h \downarrow 0}\frac{S_h-I}{h}\int_{0}^{\infty} e^{-\lambda t}S_t u\diff t=AT_{\lambda}u
    \end{align*}
Then we have $T_{\lambda} Au=AT_{\lambda}u$. We next show the second equality of \eqref{eq:rvt}.
   \begin{align*}
   AT_{\lambda}u&
   =\lim_{h \downarrow 0}\frac{S_h-I}{h}\int_{0}^{\infty} e^{-\lambda t}S_t u\diff t=\lim_{h \downarrow 0}(\frac{1}{h}\int_{0}^{\infty} e^{-\lambda t}S_{t+h} u\diff t)-\frac{1}{h}\int_{0}^{\infty} e^{-\lambda t}S_t u\diff t)\\
   &=\lim_{h \downarrow 0}(\frac{1}{h}\int_{h}^{\infty} e^{-\lambda( t-h)}S_{t} u\diff t)-\frac{1}{h}\int_{0}^{\infty} e^{-\lambda t}S_t u\diff t)\\
   &=\lim_{h \downarrow 0}- e^{\lambda h}\frac{1}{h}\int_{0}^{h} e^{-\lambda t}S_t u\diff t+\lim_{h \downarrow 0}\frac{1}{h}(e^{\lambda h}-1)\int_{0}^{\infty} e^{-\lambda t}S_t u\diff t=-u+\lambda T_{\lambda}u
\end{align*}
Thus, statement (ii) holds and statement (i) and (iii) follow from the proof of (ii).
\end{proof}
\subsection{Examples}
\subsubsection{Diffusion Semigroups(One-Dimenstional).} Consider the Banach Space $E=C[0,1]$ and the differential operator 
\begin{equation*}
Af:=f''
\end{equation*}  
with the domain $D(A)\coloneq\{f\in C^2[0,1]:f'(0)=f'(1)=0\}.$ Now we define $(e_n)$ as
\begin{equation*}
s\mapsto e_n(s) :=\begin{cases}1 & \text{if $n =0$} \\
                                                   \sqrt{2} \cos(\pi n s) & \text{if $n\geq 1$}
                             \end{cases}
\end{equation*}
It is obvious that $(e_n)\subset D(A).$
\begin{lemma}
 $Y\coloneq \emph{lin}\{e_n:n\geq0\}$ is a dense algebra of $E.$
\end{lemma}
\begin{proof}
	By Stone–Weierstrass theorem,  The set of all polynomial functions forms a  dense subalgebra of $E$. That is,
	\begin{equation}\label{eq:poly}
	\text{lin}\{f_n: f_n(s)=s^n, s\in[0,1], n\geq0\}
	\end{equation} is a dense subalgebra of $E.$ We now let $x=\cos(\pi s)$ with $s\in[0,1]$ and so $x\in[0,1]$. In this way, we replace $s$ in \eqref{eq:poly} by $\cos(\pi s)$ and then
	\begin{equation*}
	Y'\coloneq\text{lin}\{f_n: f_n(s)=\cos(\pi s)^n, s\in[0,1], n\geq0\}
	\end{equation*} is a dense subalgebra of $E.$ 
	
	By trigonometric identities, we have
	\begin{equation}\label{eq:trigo}
	\cos(\alpha+\beta)+\cos(\alpha-\beta)=2\cos(\alpha)\cos(\beta)
	\end{equation}
	Thus, by induction, $\cos(\pi s)^n\in Y$ for all $n$ and so $ Y'\subset Y$ which finishes the proof.
\end{proof}
As we know that $D(A)$ is a linear space containing $(e_n)$, $Y\subset D(A)$ and so $D(A)$ is a dense subspace of $E.$ Define the \navy{rank-one operators} as
\begin{equation*}
e_n\otimes e_n\colon f\mapsto \big \langle f,e_n \big \rangle\ e_n\coloneq \Big (\int_{0}^{1}f(s)e_n(s)\diff s\Big)\ e_n
\end{equation*}
where $\|e_n\otimes e_n\| \leq 2$ and  $(e_n\otimes e_n)\ e_m=\mathbf{1}_{\{m=n\}}\ e_n$ for all $m,n$. The operators $(S_t)$ is defined as 
\begin{equation}\label{eq:diff-semi}
	S_t=\sum_{n=0}^{\infty} e^{-\pi^2n^2 t} \cdot e_n\otimes e_n
\end{equation}
Then we have for $f\in E$,
\begin{align*}
	S_tf&=\sum_{n=0}^{\infty} e^{-\pi^2n^2 t} \cdot \langle f,e_n \big \rangle\ e_n\\
	       &=\sum_{n=0}^{\infty} e^{-\pi^2n^2 t} \cdot \Big (\int_{0}^{1}f(s)e_n(s)\diff s\Big)\ e_n
\end{align*}
For $r\in [0,1]$, 
\begin{align*}
	(S_tf)(r)&=\sum_{n=0}^{\infty} e^{-\pi^2n^2 t} \cdot \Big (\int_{0}^{1}f(s)e_n(s)\diff s\Big)\ e_n(r)\\
	               &=\sum_{n=0}^{\infty} e^{-\pi^2n^2 t} \cdot \Big (\int_{0}^{1}f(s)e_n(s)e_n(r)\diff s\Big)\ \\
	               &=\int_{0}^{1}f(s)\sum_{n=0}^{\infty} e^{-\pi^2n^2 t} \cdot e_n(s)e_n(r)\diff s \\
	               &=\int_{0}^{1}f(s)\Big(2\sum_{n=1}^{\infty} e^{-\pi^2n^2 t} \cdot \cos(n\pi s)\cos(n\pi r)+1\Big)\diff s 	               
\end{align*}
We denote $k_t(s,r)=2\sum_{n=1}^{\infty} e^{-\pi^2n^2 t} \cdot \cos(n\pi s)\cos(n\pi r)+1$, and $(S_tf)(r)=\int_{0}^{1}f(s) k_t(s,r)\diff s $. For all $S_t$,
\begin{equation*}
	\|S_t\|=\sup_{f\in E}\frac{\|S_t f\|}{\|f\|}=\sup_{f\in E}\frac{\sup_{r\in [0,1]}\|\int_{0}^{1}f(s) k_t(s,r)\diff s \|}{\|f\|}\leq \sup_{r\in [0,1]}\|\int_{0}^{1}k_t(s,r)\diff s \|=1
\end{equation*}
and since $\|S_t\1\|=1$, $\|S_t\|=1.$

The Jacobi identity is 
\begin{equation*}
	w_t(s)=\frac{1}{\sqrt{4\pi t}}\sum_{\mathbb{Z}}e^{-(s+2n)^2/4t}=\frac{1}{2}+\sum_{n=1}^{\infty} e^{-\pi^2n^2 t} \cdot \cos(n\pi s)
\end{equation*}
Applying \eqref{eq:trigo}, $k_t(s,r)=w_t(s+r)+w_t(s-r)$. 
\begin{proposition}
	$(S_t)$ with $S_0=I$ is a $C_0$ semigroup on E and its generator is given by
	\begin{equation*}
		Af=f''
	\end{equation*}
   with the domain $D(A)\coloneq\{f\in C^2[0,1]:f'(0)=f'(1)=0\}.$
\end{proposition}
\begin{proof}
	Note that the property of rank-one operators $(e_n\otimes e_n)\ e_m=\mathbf{1}_{\{m=n\}}\ e_n$ for all $m,n$. It is obvious that $S_{t+s}=S_tS_t$ holds on $Y$ and so on $E$ by continuity. Similarly, $t\mapsto S_t u$ is continuous for all $u\in E.$ 
	
	We next show the generator of $B$ of $(S_t)$ coincides with $A$ defined above. By simple calculations, $Bf=Af$ for all $f\in Y.$ By definition of $A$ and Wiki\footnote{\url{https://en.wikipedia.org/wiki/Spectral_theory_of_ordinary_differential_equations}}, $1\in \rho(A)$. Theorem 1.10(ii) of \cite{engel2006short} implies $1\in \rho(B)$ as well. Thus, $(A-I)^{-1}$ coincides with $(B-I)^{-1}$ on $B(Y)$ which is dense in $E.$ Boundeness of $(A-I)^{-1}$ and $(B-I)^{-1}$ implies $(A-I)^{-1}$ and $(B-I)^{-1}$ agree on $E$ and so $A=B.$
\end{proof}
\newpage
\bibliographystyle{apalike}
\bibliography{sgs_bib}


\end{document}

