\documentclass[12pt, reqno]{amsart}


\usepackage{amsmath, amssymb, amsthm, amsfonts}

% avoid `Too many math alphabets used in version normal` error
\newcommand\hmmax{0}
\newcommand\bmmax{0}

\usepackage[mathscr]{euscript} 
\usepackage{stmaryrd} % St Mary's Road symbols font --- some extra symbols

%\usepackage{fontspec} 
\usepackage[xcharter]{newtxmath}
%\setmainfont{XCharter}

\usepackage{graphics, stackrel}
\usepackage{graphicx}

\usepackage{verbatim}
\usepackage{natbib}
\usepackage{enumitem}

%font
%\usepackage{lmodern}
%\usepackage[T1]{fontenc}
%\usepackage{fontspec} 
%\usepackage[xcharter]{newtxmath}
%\setmainfont{XCharter}
%\usepackage{mathpazo}
%\usepackage{tgpagella}

%subfloats / figures
\usepackage{caption}
\usepackage{subcaption}

% For pandas latex tables
\usepackage{booktabs}


\usepackage{fancyvrb}
\usepackage[dvipsnames,svgnames,table]{xcolor}
\usepackage{mdwlist}

\usepackage[breaklinks=true, citecolor=Brown, colorlinks=true, linkcolor=blue]{hyperref}

% lists
\usepackage{enumitem}
\setlist[enumerate]{itemsep=2pt,topsep=3pt}
\setlist[itemize]{itemsep=2pt,topsep=3pt}
\setlist[enumerate,1]{label=(\roman*)}

\usepackage{mathrsfs}  % caligraphic
%\usepackage{stix} 
\usepackage{bbm}
\usepackage{bm}        % bold symbols


%% page layout
\usepackage[left=1.25in, right=1.25in, top=1.0in, bottom=1.15in, includehead, includefoot]{geometry}

% nice inequalities
\renewcommand{\leq}{\leqslant}
\renewcommand{\geq}{\geqslant}

% inner product
\providecommand{\inner}[1]{\left\langle{#1}\right\rangle}
\providecommand{\innerp}[1]{\left\langle{#1}\right\rangle_\pi}

% lists
\usepackage{enumitem}
\setlist[enumerate]{itemsep=2pt,topsep=3pt}
\setlist[itemize]{itemsep=2pt,topsep=3pt}
\setlist[enumerate,1]{label={\upshape (\roman*)}}


\usepackage[ruled, linesnumbered]{algorithm2e}

%extra spacing
\renewcommand{\baselinestretch}{1.25}

%horizonal line
\newcommand{\HRule}{\rule{\linewidth}{0.3mm}}

% skip a line between paragraphs, no indentation
%\setlength{\parskip}{1.5ex plus0.5ex minus0.5ex}
%\setlength{\parindent}{0pt}

% footnote without a maker (blfootnote)
\newcommand\blfootnote[1]{%
  \begingroup
  \renewcommand\thefootnote{}\footnote{#1}%
  \addtocounter{footnote}{-1}%
  \endgroup
}

\DeclareMathOperator{\fix}{fix}
\DeclareMathOperator{\Span}{span}
\DeclareMathOperator{\diag}{diag}
\DeclareMathOperator*{\argmin}{arg\,min}
\DeclareMathOperator*{\argmax}{arg\,max}

\DeclareMathOperator{\cl}{cl}
\DeclareMathOperator{\Int}{int}
%\DeclareMathOperator{\overset{\circ}}{int}
\DeclareMathOperator{\Prob}{Prob}
\DeclareMathOperator{\determinant}{det}
\DeclareMathOperator{\Var}{Var}
\DeclareMathOperator{\Cov}{Cov}
\DeclareMathOperator{\graph}{graph}

% mics short cuts and symbols
\newcommand{\st}{\ensuremath{\ \mathrm{s.t.}\ }}
\newcommand{\setntn}[2]{ \{ #1 : #2 \} }
\newcommand{\fore}{\therefore \quad}
\newcommand{\preqsd}{\preceq_{sd} }
\newcommand{\toas}{\stackrel {\textrm{ \scriptsize{a.s.} }} {\to} }
\newcommand{\tod}{\stackrel { d } {\to} }
\newcommand{\too}{\stackrel { o } {\to} }
\newcommand{\toweak}{\stackrel { w } {\to} }
\newcommand{\topr}{\stackrel { p } {\to} }
\newcommand{\disteq}{\stackrel { \mathscr D } {=} }
\newcommand{\eqdist}{\stackrel {\textrm{ \scriptsize{d} }} {=} }
\newcommand{\iidsim}{\stackrel {\textrm{ {\sc iid }}} {\sim} }
\newcommand{\1}{\mathbbm 1}
\newcommand{\la}{\langle}
\newcommand{\ra}{\rangle}
\newcommand{\dee}{\,{\rm d}}
\newcommand{\og}{{\mathbbm G}}
\newcommand{\ctimes}{\! \times \!}
\newcommand{\sint}{{\textstyle\int}}

\newcommand{\given}{\, | \,}
\newcommand{\A}{\forall}

% d for integrals
\newcommand*\diff{\mathop{}\!\mathrm{d}}
\newcommand*\e{\mathrm{e}}

% nice emptyset
\let\oldemptyset\emptyset
\let\emptyset\varnothing


%\renewcommand{\times}{\! \times \!}

\newcommand{\aA}{\mathcal A}
\newcommand{\cC}{\mathscr C}
\newcommand{\sS}{\mathcal S}
\newcommand{\bB}{\mathcal B}
\newcommand{\oO}{\mathcal O}
\newcommand{\gG}{\mathcal G}
\newcommand{\hH}{\mathcal H}
\newcommand{\kK}{\mathcal K}
\newcommand{\iI}{\mathcal I}
\newcommand{\eE}{\mathcal E}
\newcommand{\fF}{\mathscr F}
\newcommand{\qQ}{\mathcal Q}
\newcommand{\tT}{\mathcal T}
\newcommand{\xX}{\mathcal X}
\newcommand{\yY}{\mathcal Y}
\newcommand{\rR}{\mathcal R}
\newcommand{\zZ}{\mathcal Z}
\newcommand{\wW}{\mathcal W}
\newcommand{\uU}{\mathcal U}
\newcommand{\lL}{\mathcal L}
\newcommand{\mM}{\mathcal M}
\newcommand{\dD}{\mathcal D}
\newcommand{\pP}{\mathcal P}
\newcommand{\vV}{\mathcal V}

\newcommand{\Bsf}{\mathsf B}
\newcommand{\Hsf}{\mathsf H}
\newcommand{\Vsf}{\mathsf V}

\newcommand{\BB}{\mathbbm B}
\newcommand{\DD}{\mathbbm D}
\newcommand{\RR}{\mathbbm R}
\newcommand{\CC}{\mathbbm C}
\newcommand{\QQ}{\mathbbm Q}
\newcommand{\NN}{\mathbbm N}
\newcommand{\GG}{\mathbbm G}
\newcommand{\UU}{\mathbbm U}
\newcommand{\TT}{\mathbbm T}
\newcommand{\YY}{\mathbbm Y}
\newcommand{\ZZ}{\mathbbm Z}
\newcommand{\HH}{\mathbbm H}
\newcommand{\MM}{\mathbbm M}
\newcommand{\PP}{\mathbbm P}
\newcommand{\VV}{\mathbbm V}
\newcommand{\EE}{\mathbbm E}


\newcommand{\bH}{\mathbf H}
\newcommand{\bT}{\mathbf T}

\newcommand{\var}{\mathbbm V}

\newcommand{\Xsf}{\mathsf X}
\newcommand{\Msf}{\mathsf M}
\newcommand{\Asf}{\mathsf A}
\newcommand{\Gsf}{\mathsf G}
\newcommand{\II}{\mathsf I}
\newcommand{\WW}{\mathsf W}

\renewcommand{\phi}{\varphi}
\renewcommand{\epsilon}{\varepsilon}

\newcommand{\bP}{\mathbf P}
\newcommand{\bQ}{\mathbf Q}
\newcommand{\bE}{\mathbf E}
\newcommand{\bM}{\mathbf M}
\newcommand{\bX}{\mathbf X}
\newcommand{\bY}{\mathbf Y}

\theoremstyle{plain}
\newtheorem{theorem}{Theorem}[section]
\newtheorem{corollary}[theorem]{Corollary}
\newtheorem{lemma}[theorem]{Lemma}
\newtheorem{proposition}[theorem]{Proposition}


\theoremstyle{definition}
\newtheorem{definition}{Definition}[section]
\newtheorem{axiom}{Axiom}[section]
\newtheorem{example}{Example}[section]
\newtheorem{remark}{Remark}[section]
\newtheorem{notation}{Notation}[section]
\newtheorem{assumption}{Assumption}[section]
\newtheorem{condition}{Condition}[section]


%\DeclareTextFontCommand{\emph}{\bfseries}

%%%%%%%%%%%%%%%%%% end my preamble %%%%%%%%%%%%%%%%%%%%%%%%%%%%%%%%%%%


\newcommand{\navy}[1]{\textcolor{blue}{#1}}




\begin{document}


\title{Semigroup Notes}

\author{JS and JY}

\date{\today}

\maketitle

\section{Definitions}

Let $E$ be any set and let $(S_t) \coloneq (S_t)_{t \geq 0}$ be a family of
self-maps on $E$ with index $t \in \RR_+$. The pair $(E, (S_t))$ is called a \navy{semidynamical system} if 
$S_0$ is the idendity and $(S_t)$ has the semigroup property
%
\begin{equation*}
    S_{s + t} = S_t \circ S_s
    \quad \text{for all } s, t \in \RR_+.
\end{equation*}
%
If $E$ is a vector space and each $S_t$ is linear, then $(E, (S_t))$ is
called an \navy{algebraic operator (AO) semigroup}.  If, in addition, $E$ is an
Banach space and $t \mapsto S_t u$ is continuous for all $u \in E$, then $(E,
(S_t))$
is called a \navy{$C_0$-semigroup}.  When $E$ is understood, we say that $(S_t)$
is a $C_0$-semigroup.



\section{Continuity results}

In what follows, $E$ is a Banach space and $\lL(E)$ is the set of bounded linear
operators from $E$ to itself.  The symbol $\| \cdot \|$ denotes either the norm
on $E$ or the operator norm on $\lL(E)$, depending on context.

Let $K$ be a compact subset of $\RR$ and let $\{S_t\}_{t \in K}$ be a subset of
$\lL(E)$.  The following result is from \cite{engel2006short}.

\begin{lemma}\label{l:contin}
    The following statements are equivalent:
    %
    \begin{enumerate}
        \item The map $t \mapsto S_t u$ is continuous on $K$ for all $u \in E$.
        \item $\|S_t\|$ is bounded over $t \in K$ and there exists a dense subset
            $D$ of $E$ such that $t \mapsto S_t u$ is continuous on $K$ for all $u \in D$.
        \item For any compact $C \subset E$, the map $(t, u) \mapsto S_t u$ is
            uniformly continuous on $K \times C$.
    \end{enumerate}
\end{lemma}

\begin{proof}
    ((i) $\implies$ (ii)) By (i), for any $u \in E$, the map $t \mapsto S_t u$ is
    continuous on a compact set and, therefore, its image is bounded in $E$.  
    Hence, by the uniform boundedness principle, $\|S_t\|$ is bounded over $t
    \in K$.  The statement in (ii) regarding continuity is obvious.

    ((ii) $\implies$ (iii)).  Fix compact $C \subset E$ and $\epsilon > 0$.
    We metrize $K \times C$ by setting $d((s, u), (t, v)) = \| u - v\| \vee |s-t|$.
    Choose $M \in \NN$ such that $\|S_t\| \leq M$ for all $t \in K$.  Let $D$ be the
    dense set in (ii) and observe that the set of open balls $B(u, \epsilon/M)$
    over $u \in D$ provides an open cover of $C$.  As such, we can choose a finite
    set $D_F \subset D$ such that $C$ is contained in $\cup_{u \in D_F} B(u,
    \epsilon/M)$. Since, for each $u \in D_F$, the map
    $t \mapsto S_t u$ is continuous on a compact set, it is also uniformly
    continuous.  As a result, given $u \in D_F$, we can select a $\delta_u > 0$ such that
    %
    \begin{equation*}
        |s - t| < \delta_u \implies \| S_s u - S_t u \| < \epsilon.
    \end{equation*}
    %
    Let $\delta$ be the minimum of $\{\delta_u\}_{u \in D_F}$ and $\epsilon / M$.
    If we take $u, v \in C$ and $s, t \in K$ with $d((s, u), (t, v)) < \delta$, then,
    choosing $w \in D_F$ with $\|u - w \| < \epsilon / M$, we have
    %
    \begin{align*}
        \| S_s u - S_t v \|
        & \leq \| S_s u - S_s w \| 
            + \| S_s w - S_t w \| 
                + \| S_t w - S_t v \|
                \\
        & = \| S_s (u - w) \| 
            + \| S_s w - S_t w \| 
                + \| S_t (w - v) \|
                \\
        & < M (\epsilon / M) + \epsilon + M (2 \epsilon / M) = 4 \epsilon.
    \end{align*}
    %
    Hence $(t, u) \mapsto S_t u$ is
            uniformly continuous on $K \times C$, as claimed.

    ((iii) $\implies$ (i)) This implication is trivial (take $C$ to be a
    singleton).
\end{proof}

\begin{lemma}\label{l:ubfc}
    If $(S_t)_{t \geq 0}$ is a $C_0$-semigroup on $E$, then
        $\sup_{t \leq \delta} \| S_t \| < \infty$ for all $\delta > 0$.
\end{lemma}

\begin{proof}
    We first claim there exists an $\epsilon > 0$ such that $
    \sup_{t \leq \epsilon} \| S_t \| < \infty$.  Indeed, if no such $\epsilon$
    exists, then there exists a sequence $t_n \to 0$ such that 
    $\|S_{t_n}\|$ is unbounded.  But then, by the principle of uniform
    boundedness, there exists a $u \in E$ such that $\|S_{t_n} u\|$ is
    unbounded.  This contradicts the continuity property of $C_0$-semigroups.

    Now let $\epsilon$ be as above and choose $M \in \NN$ with $\| S_t \| \leq
    M$ whenever $t \leq \epsilon$.  Fix $k \in \NN$ and $t \leq k \epsilon$.
    Since $S_t$ is $k$ compositions of $S_{t/k}$, and since $t/k < \epsilon$, 
    the semigroup property yields $\| S_t \| \leq k M$.  Hence $t \mapsto S_t$
    is bounded on $[0, k \epsilon]$.  Since $k$ was an arbitrary element of
    $\NN$, this proves the claim in Lemma~\ref{l:ubfc}.
\end{proof}

\begin{lemma}\label{l:semiequivcon}
    An AO semigroup $(S_t)_{t \geq 0}$ on $E$ is a 
    $C_0$-semigroup on $E$ if and only if $\lim_{t \downarrow 0} S_t u = u$ for all $u \in E$.
\end{lemma}

\begin{proof}
    Sufficiency is obvious.  Regarding necessity, fix $u \in E$ and $t > 0$.  We
    need to show that $\|S_{t+h} u - S_t u\| \to 0$ as $h \to 0$.  Suppose first
    that $h \downarrow 0$.  Then
    %
    \begin{equation*}
        \|S_{t+h} u - S_t u\|  
        = \|S_t S_h u - S_t u\|  
        \leq \|S_t \| \| S_h u - u\|  \to 0.
    \end{equation*}
    %
    If, on the other hand $h \uparrow 0$, then
    %
    \begin{equation*}
        \|S_{t+h} u - S_t u\|  
        = \|S_{t+h} u - S_{t + h} S_{-h} u\|  
        \leq \|S_{t+h} \| \| u - S_{-h} u\| \to 0 .
    \end{equation*}
    %
    In the last step we used the fact that $\|S_{t+h} \| $ is bounded over $h$
    by Lemma~\ref{l:ubfc}.
\end{proof}

\begin{lemma}\label{l:aose}
    Let $(S_t)_{t \geq 0}$ be an AO semigroup on $E$.
    If there exists a dense subset $D$ of $E$ such that $\lim_{t \downarrow 0}
    S_t u = u$ for all $u \in D$ and, in addition, $\sup_{t \leq \delta} \|S_t
    \| < \infty$ for some $\delta > 0$, then $(S_t)_{t \geq 0}$ is a
    $C_0$-semigroup.
\end{lemma}

\begin{proof}
    Fix $u \in E$.  By Lemma~\ref{l:semiequivcon} it suffices to show
    that, for a given sequence $t_n \downarrow 0$, we have $S_{t_n} u \to u$ as
    $n \to 0$. 
    To see that this holds, fix $t_n \downarrow 0$ and choose a compact subset
    $K$ of $\RR_+$ such
    that $\{t_n\} \subset K$.  Since $K$ is compact, $K \ni t \mapsto S_t w$ is
    continuous when $w \in D$, and $\|S_t \|$ is bounded over $t \in K$,
    Lemma~\ref{l:contin} implies that $K \ni t \mapsto S_t u$ is continuous.
    In particular, $S_{t_n} u \to u$ as $n \to 0$. 
\end{proof}



\section{Examples}


\subsection{Left-shift semigroups}

Let $C_0(\RR_+)$ be the set of all continuous real-valued functions $f$ on
$\RR_+$ with $f(x) \to 0$ as $x \to \infty$.  The set $C_0(\RR_+)$ is paired
with the supremum norm.  Consider the \navy{left
translation semigroup} given by $(S^\ell_t f)(x) = f(x + t)$.

\begin{lemma}\label{l:ltsemi}
    $(S^\ell_t)$ is a $C_0$-semigroup on $C_0(\RR_+)$.  
\end{lemma}

\begin{proof}
    Evidently $S^\ell_0 f = f$.  The semigroup property holds because, for $s, t
    \geq 0$, we have
    %
    \begin{equation*}
        (S^\ell_{s + t} f)(x) = f(x + s + t) = (S^\ell_t (S^\ell_s f))(x).
    \end{equation*}
    %
    Regarding continuity, fix $f \in C_0(\RR_+)$ and let $(t_n)$ be a real sequence
    with $t_n \downarrow 0$.  Fix $\epsilon > 0$.  Since $f$ is uniformly
    continuous, we can select a $\delta > 0$ such that $|f(x) - f(y)| < \epsilon$
    whenever $|x-y|<\delta$.  Let $N \in \NN$ be such that $t_n < \delta$ when
    $n \geq N$.  Then, for $n \geq N$,
    %
    \begin{equation*}
        \| S^\ell_{t_n} f - f \|
        = \sup_x | f(x + t_n)  - f(x) |
        < \epsilon.
    \end{equation*}
    %
    Hence $S^\ell_t f \downarrow f$ and $(S^\ell_t)$ is a $C_0$-semigroup.
\end{proof}


Let $C_0^1(\RR_+)$ be the set of all continuously differentiable $f \in
C_0(\RR+)$ with $f' \in C_0(\RR+)$.  The set $C_0^1(\RR_+)$ is paired
with the norm $\|f\| = \sup_x |f(x)| + \sup_x |f'(x)|$.  

\begin{lemma}\label{l:ltsemi2}
    $(S^\ell_t)$ is a $C_0$-semigroup on $C_0^1(\RR_+)$.  
\end{lemma}

\begin{proof}
    In view of Lemma~\ref{l:ltsemi}, we only need to check continuity.
    Fixing $f \in C_0^1(\RR_+)$, we have
    %
    \begin{equation*}
        \| S^\ell_t f - f \|
        = \sup_x | f(x + t)  - f(x) | + \sup_x | f'(x + t)  - f'(x) |
    \end{equation*}
    %
    Since $f$ and $f'$ are both in $C_0(\RR_+)$, the proof of
    Lemma~\ref{l:ltsemi} implies that both terms on the right hand side
    converge to zero as $t \downarrow 0$.  Hence continuity holds.
\end{proof}

\subsection{Right-shift semigroups}\label{ss:rssemi}

Here we discuss right-shift semigroups.  We will embed them in a space of
integrable functions.  Below $\lambda$ denotes Lebesgue measure.

Let $C_c(\RR)$ be the set of all continuous real-valued functions $f$ on
$\RR$ that vanish off a compact set.  Let $L_1(\RR)$ be the set of 
Borel measurable real-valued functions on $\RR$ with $\| f \| \coloneq \int
|f| \diff \lambda < \infty$.  Let $S_t$ be the linear operator on $L_1(\RR)$
defined by 
%
\begin{equation*}
    (S_t f)(x) = f(x + t)
    \qquad (x \in \RR, \; t \geq 0).
\end{equation*}
%

\begin{lemma}\label{l:rtsemi}
    $(S^\ell_t)$ is a $C_0$-semigroup on $L_1(\RR)$.  
\end{lemma}

\begin{proof}
    It is simple to confirm that $(S_t)$ is an AO semigroup on $L_1(\RR)$.
    Regarding continuity, note that $\|S_t f\| = \int |f(x-t)| \diff x = \| f\|$, so
    $\|S_t\|$ is bounded in $t$.  Since $C_c(\RR)$ is dense in $L_1(\RR)$ under this
    norm, Lemma~\ref{l:aose} implies that, to show $(S_t)$ is a $C_0$-semigroup
    on $L_1(\RR)$, it suffices to show that $\| S_t f - f \| \to 0$ for any $f \in
    C_c(\RR)$.

    To this end, fix $f \in C_c(\RR)$ and let $K$ be a compact set such that $f$
    vanishes off $K$.  Fix $\epsilon > 0$.  By uniform continuity, we can take a
    $\delta > 0$ such that $|x-y| < \delta$ implies $|f(x) - f(y)| < \epsilon /
    \lambda(K)$.  If $t < \delta$, then
    %
    \begin{equation*}
        \| S_t f - f \| 
        = \int | f(x-t) - f(x) | \diff x
        \leq \lambda(K) \frac{\epsilon}{\lambda(K)} = \epsilon.
    \end{equation*}
    %
    This completes the proof of $C_0$-continuity of $(S_t)$ on $L_1(\RR)$.
\end{proof}


\subsection{Multiplication semigroups}\label{ss:multisemi}

Let $(\Xsf, \bB, \mu)$ be a $\sigma$-finite measure space and let
$\phi$ be a measurable map from $\Xsf$ to $\RR_+$.  Define
%
\begin{equation*}
    S_t f = \exp(-t \phi(x)) f(x) 
    \qquad (x \in \Xsf, \; t \geq 0).
\end{equation*}
%
The family $(S_t)$ is called a \navy{multiplication semigroup}.

\begin{lemma}\label{l:multisemi}
    $(S_t)$ is a $C_0$-semigroup on $L_1(\Xsf, \bB, \mu)$.
\end{lemma}

\begin{proof}
    It is simple to confirm that $(S_t)$ is an AO semigroup on $L_1(\RR)$.
    Regarding continuity, fix $f \in L_1(\Xsf, \bB, \mu)$ and observe that
    %
    \begin{equation*}
        \| S_t f - f \|
        = \int |f(x)| | \exp(-t\phi(x)) - 1 | \mu(\diff x).
    \end{equation*}
    %
    It follows from the dominated convergence theorem that this integral
    converges to zero as $t \downarrow 0$.  Hence $(S_t)$ is a $C_0$-semigroup
    on $L_1(\Xsf, \bB, \mu)$.
\end{proof}



\subsection{Uniformly continuous semigroups}\label{ss:ucsemi}

Let $E$ be a Banach space and let $\lL(E)$ be the bounded linear operators on $E$.
Recall that the \navy{exponential} of $A \in \lL(E)$ is given by
%
\begin{equation*}
    \exp(A) \coloneq \sum_{n=0}^\infty \frac{A^n}{n!}.
\end{equation*}
%
Fixing $A \in \lL(E)$, consider the family of linear operators on $E$ given
by 
%
\begin{equation*}
    S_t u = \exp(t A) u
    \qquad (u \in E, \; t \geq 0)
\end{equation*}
%
We recall that the exponential function $\phi(t) \coloneq \exp(t A)$ 
%
\begin{enumerate}
    \item obeys $\phi(0) = I$ and $\phi(s + t) = \phi(t) \phi(s)$ for all $s, t
        \in \RR$; and
    \item is continuous as a map from $\RR$ to $\lL(E)$.
\end{enumerate}
%
From (i) we can easily confirm that $(S_t)$ is an algebraic operator semigroup
on $E$.  Regarding continuity, (i) and (ii) imply that
%
\begin{equation}\label{eq:ucsemi}
    \lim_{t \downarrow 0} \| S_t - I \| = 0.
\end{equation}
%
It follows from \eqref{eq:ucsemi} that $(S_t)$ is a $C_0$-semigroup on $E$.

Any operator semigroup $(S_t)$ on $E$ obeying \eqref{eq:ucsemi} is called a
\navy{uniformly continuous semigroup}.  In fact no other examples exist:

\begin{theorem}\label{t:ucsemi}
    If $(S_t)$ is a uniformly continuous semigroup on $E$, then there exists an
    $A \in \lL(E)$ such that $S_t u = \exp(t A) u$ for all $u \in E$ and $t \geq 0$.
\end{theorem}

The proof of Theorem~\ref{t:ucsemi} can be found 2(b) of \cite{engel2006short}.

\bibliographystyle{apalike}
\bibliography{sgs_bib}


\end{document}

