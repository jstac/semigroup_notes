\documentclass[12pt, reqno]{amsart}

\usepackage{amsmath, amssymb, amsthm, amsfonts}

% avoid `Too many math alphabets used in version normal` error
\newcommand\hmmax{0}
\newcommand\bmmax{0}

\usepackage[mathscr]{euscript} 
\usepackage{stmaryrd} % St Mary's Road symbols font --- some extra symbols

%\usepackage{fontspec} 
\usepackage[xcharter]{newtxmath}
%\setmainfont{XCharter}

\usepackage{graphics, stackrel}
\usepackage{graphicx}

\usepackage{verbatim}
\usepackage{natbib}
\usepackage{enumitem}

%font
%\usepackage{lmodern}
%\usepackage[T1]{fontenc}
%\usepackage{fontspec} 
%\usepackage[xcharter]{newtxmath}
%\setmainfont{XCharter}
%\usepackage{mathpazo}
%\usepackage{tgpagella}

%subfloats / figures
\usepackage{caption}
\usepackage{subcaption}

% For pandas latex tables
\usepackage{booktabs}


\usepackage{fancyvrb}
\usepackage[dvipsnames,svgnames,table]{xcolor}
\usepackage{mdwlist}

\usepackage[breaklinks=true, citecolor=Brown, colorlinks=true, linkcolor=blue]{hyperref}

% lists
\usepackage{enumitem}
\setlist[enumerate]{itemsep=2pt,topsep=3pt}
\setlist[itemize]{itemsep=2pt,topsep=3pt}
\setlist[enumerate,1]{label=(\roman*)}

\usepackage{mathrsfs}  % caligraphic
%\usepackage{stix} 
\usepackage{bbm}
\usepackage{bm}        % bold symbols


%% page layout
\usepackage[left=1.25in, right=1.25in, top=1.0in, bottom=1.15in, includehead, includefoot]{geometry}

% nice inequalities
\renewcommand{\leq}{\leqslant}
\renewcommand{\geq}{\geqslant}

% inner product
\providecommand{\inner}[1]{\left\langle{#1}\right\rangle}
\providecommand{\innerp}[1]{\left\langle{#1}\right\rangle_\pi}

% lists
\usepackage{enumitem}
\setlist[enumerate]{itemsep=2pt,topsep=3pt}
\setlist[itemize]{itemsep=2pt,topsep=3pt}
\setlist[enumerate,1]{label={\upshape (\roman*)}}


\usepackage[ruled, linesnumbered]{algorithm2e}

%extra spacing
\renewcommand{\baselinestretch}{1.25}

%horizonal line
\newcommand{\HRule}{\rule{\linewidth}{0.3mm}}

% skip a line between paragraphs, no indentation
%\setlength{\parskip}{1.5ex plus0.5ex minus0.5ex}
%\setlength{\parindent}{0pt}

% footnote without a maker (blfootnote)
\newcommand\blfootnote[1]{%
  \begingroup
  \renewcommand\thefootnote{}\footnote{#1}%
  \addtocounter{footnote}{-1}%
  \endgroup
}

\DeclareMathOperator{\fix}{fix}
\DeclareMathOperator{\Span}{span}
\DeclareMathOperator{\diag}{diag}
\DeclareMathOperator*{\argmin}{arg\,min}
\DeclareMathOperator*{\argmax}{arg\,max}

\DeclareMathOperator{\cl}{cl}
\DeclareMathOperator{\Int}{int}
%\DeclareMathOperator{\overset{\circ}}{int}
\DeclareMathOperator{\Prob}{Prob}
\DeclareMathOperator{\determinant}{det}
\DeclareMathOperator{\Var}{Var}
\DeclareMathOperator{\Cov}{Cov}
\DeclareMathOperator{\graph}{graph}

% mics short cuts and symbols
\newcommand{\st}{\ensuremath{\ \mathrm{s.t.}\ }}
\newcommand{\setntn}[2]{ \{ #1 : #2 \} }
\newcommand{\fore}{\therefore \quad}
\newcommand{\preqsd}{\preceq_{sd} }
\newcommand{\toas}{\stackrel {\textrm{ \scriptsize{a.s.} }} {\to} }
\newcommand{\tod}{\stackrel { d } {\to} }
\newcommand{\too}{\stackrel { o } {\to} }
\newcommand{\toweak}{\stackrel { w } {\to} }
\newcommand{\topr}{\stackrel { p } {\to} }
\newcommand{\disteq}{\stackrel { \mathscr D } {=} }
\newcommand{\eqdist}{\stackrel {\textrm{ \scriptsize{d} }} {=} }
\newcommand{\iidsim}{\stackrel {\textrm{ {\sc iid }}} {\sim} }
\newcommand{\1}{\mathbbm 1}
\newcommand{\la}{\langle}
\newcommand{\ra}{\rangle}
\newcommand{\dee}{\,{\rm d}}
\newcommand{\og}{{\mathbbm G}}
\newcommand{\ctimes}{\! \times \!}
\newcommand{\sint}{{\textstyle\int}}

\newcommand{\given}{\, | \,}
\newcommand{\A}{\forall}

% d for integrals
\newcommand*\diff{\mathop{}\!\mathrm{d}}
\newcommand*\e{\mathrm{e}}

% nice emptyset
\let\oldemptyset\emptyset
\let\emptyset\varnothing


%\renewcommand{\times}{\! \times \!}

\newcommand{\aA}{\mathcal A}
\newcommand{\cC}{\mathscr C}
\newcommand{\sS}{\mathcal S}
\newcommand{\bB}{\mathcal B}
\newcommand{\oO}{\mathcal O}
\newcommand{\gG}{\mathcal G}
\newcommand{\hH}{\mathcal H}
\newcommand{\kK}{\mathcal K}
\newcommand{\iI}{\mathcal I}
\newcommand{\eE}{\mathcal E}
\newcommand{\fF}{\mathscr F}
\newcommand{\qQ}{\mathcal Q}
\newcommand{\tT}{\mathcal T}
\newcommand{\xX}{\mathcal X}
\newcommand{\yY}{\mathcal Y}
\newcommand{\rR}{\mathcal R}
\newcommand{\zZ}{\mathcal Z}
\newcommand{\wW}{\mathcal W}
\newcommand{\uU}{\mathcal U}
\newcommand{\lL}{\mathcal L}
\newcommand{\mM}{\mathcal M}
\newcommand{\dD}{\mathcal D}
\newcommand{\pP}{\mathcal P}
\newcommand{\vV}{\mathcal V}

\newcommand{\Bsf}{\mathsf B}
\newcommand{\Hsf}{\mathsf H}
\newcommand{\Vsf}{\mathsf V}

\newcommand{\BB}{\mathbbm B}
\newcommand{\DD}{\mathbbm D}
\newcommand{\RR}{\mathbbm R}
\newcommand{\CC}{\mathbbm C}
\newcommand{\QQ}{\mathbbm Q}
\newcommand{\NN}{\mathbbm N}
\newcommand{\GG}{\mathbbm G}
\newcommand{\UU}{\mathbbm U}
\newcommand{\TT}{\mathbbm T}
\newcommand{\YY}{\mathbbm Y}
\newcommand{\ZZ}{\mathbbm Z}
\newcommand{\HH}{\mathbbm H}
\newcommand{\MM}{\mathbbm M}
\newcommand{\PP}{\mathbbm P}
\newcommand{\VV}{\mathbbm V}
\newcommand{\EE}{\mathbbm E}


\newcommand{\bH}{\mathbf H}
\newcommand{\bT}{\mathbf T}

\newcommand{\var}{\mathbbm V}

\newcommand{\Xsf}{\mathsf X}
\newcommand{\Msf}{\mathsf M}
\newcommand{\Asf}{\mathsf A}
\newcommand{\Gsf}{\mathsf G}
\newcommand{\II}{\mathsf I}
\newcommand{\WW}{\mathsf W}

\renewcommand{\phi}{\varphi}
\renewcommand{\epsilon}{\varepsilon}

\newcommand{\bP}{\mathbf P}
\newcommand{\bQ}{\mathbf Q}
\newcommand{\bE}{\mathbf E}
\newcommand{\bM}{\mathbf M}
\newcommand{\bX}{\mathbf X}
\newcommand{\bY}{\mathbf Y}

\theoremstyle{plain}
\newtheorem{theorem}{Theorem}[section]
\newtheorem{corollary}[theorem]{Corollary}
\newtheorem{lemma}[theorem]{Lemma}
\newtheorem{proposition}[theorem]{Proposition}


\theoremstyle{definition}
\newtheorem{definition}{Definition}[section]
\newtheorem{axiom}{Axiom}[section]
\newtheorem{example}{Example}[section]
\newtheorem{remark}{Remark}[section]
\newtheorem{notation}{Notation}[section]
\newtheorem{assumption}{Assumption}[section]
\newtheorem{condition}{Condition}[section]


%\DeclareTextFontCommand{\emph}{\bfseries}

%%%%%%%%%%%%%%%%%% end my preamble %%%%%%%%%%%%%%%%%%%%%%%%%%%%%%%%%%%


\newcommand{\navy}[1]{\textcolor{blue}{#1}}




\begin{document}


\title{Semigroup Notes}

\author{JS and JY}

\date{\today}

\maketitle




\section{Linear Differential Equations and Exponential Paths}

This section gives a discussion of linear ordinary differential equations (ODEs)
in Banach space when the ``vector field'' that drives the ODE is a
\emph{bounded} linear operator.  (This includes the finite-dimensional case,
since all linear operators on finite-dimensional spaces are bounded.)

These boundedness restriction leaves out many interesting and important
processes.  At the same time, it illustrates the ideal case, and some of the
properties we hope to discover for the solutions of general ODEs (driven by
possibly unbounded linear operators) in Banach space.  


\subsection{Preliminaries}\label{ss:prel}

First we recall some facts concerning calculus for Banach space valued
functions. To this end, let $u$ be a map from $\RR_+$ to $E$, where $E$ is a
Banach space. If, for some $t > 0$, the limit
%
\begin{equation*}
    \dot u(t)
    \coloneq \frac{\diff }{\diff t} u(t)
    \coloneq \lim_{h \to 0} \frac{\| u(t + h) - u(t) \|}{h}
\end{equation*}
%
exists, then we say that $u$ is \navy{differentiable} at $t$ and call $\dot u(t)$
the \navy{derivative} of $u$ at $t$.  In the case where $t=0$ we use the same
definitions after replacing $h \to 0$ with $h \downarrow 0$.
We say that $u$ is \navy{differentiable} on $\RR_+$ if $u$ is differentiable at
$t$ for all $t \in \RR_+$.


Let $u$ be a continuous map from $[a, b]$ to $E$.  We call $I(u) \in E$ the
(Riemann) \navy{integral} of $u$ over $[a,b]$ and write $I(u) = \int_a^b u(s) \diff s$ if,
for every $\epsilon > 0$, there exists  a partition 
$a = t_0 < t_1 < \cdots t_{k+1} = b$ of $[a, b]$ and points $\{s_i\}$ with
$t_{i-1} < s_i < t_i$ such that
%
\begin{equation*}
    \left\|
        \;
        \sum_{i=0}^{k+1} u(s_i) (t_i - t_{i-1})
        - I(u)
        \;
    \right\| < \epsilon
\end{equation*}
%

The next lemma states elementary properties of the integral.
(For details see, e.g., \cite{applebaum2019semigroups}.)

\begin{lemma}\label{l:riemann}
    If $u$ is continuous on $\RR_+$ and $a, b \in \RR_+$ with $a < b$, then
    %
    \begin{enumerate}
        \item $\int_a^c u(s) \diff s 
            = \int_a^b u(s) \diff s + \int_b^c u(s) \diff s$ whenever $a < b < c$,
        \item $\int_a^b u(s + c) \diff s = \int_{a+c}^{b+c} u(s)
            \diff s$ whenever $c \geq 0$,
        \item $\| \int_a^b u(s) \diff s \| \leq \int_a^b \|
            u(s)  \| \diff s$,
        \item $u(t) = \lim_{h \to 0} \, (1/h) \int_t^{t+h} u(s) \diff s$ for all
            $t \in \RR_+$, and
        \item $L \in \lL(E)$ implies $L \int_a^b u(s) \diff s = \int_a^b Lu(s) \diff s $.
    \end{enumerate}
    %
\end{lemma}

In what follows, when discussing maps such as $u$ given above,
we often write $u_t$ and $\dot u_t$ instead of $u(t)$ and $\dot u(t)$.


\subsection{Exponentials of Linear Operators}

Let $E$ be a Banach space and let $\lL(E)$ be the bounded linear operators on $E$.
The scalar exponential of a scalar $a \in \RR$ can be defined by $\exp(a)
\coloneq \sum_{n=0}^\infty \frac{a^n}{n!}$.  Analogously, the \navy{exponential}
of $A \in \lL(E)$ is given by
%
\begin{equation}\label{eq:expfun} 
    \exp(A) \coloneq \sum_{n=0}^\infty \frac{A^n}{n!}.
\end{equation}
%

\begin{lemma}\label{l:expcom}
    Let $A$ and $B$ be elements of $\lL(E)$.  The following properties hold:
    %
    \begin{enumerate}
        \item $\exp(A)$ is a well-defined element of $\lL(E)$  and $\| \exp(A) \| \leq \exp(\|A\|)$.
        \item If $A$ and $B$ commute (i.e. $AB = BA$), then $\exp(A + B) =
            \exp(A) \exp(B)$.
        \item If $m$ is any positive integer, then $\exp(mA) = (\exp(A))^m$.
    \end{enumerate}
    %
\end{lemma}

\begin{lemma}\label{l:expcom2}
    Fix $A \in \lL(E)$ and consider the function $e(t) \coloneq \exp(tA)$ from
    $\RR$ to $\lL(E)$.  The following statements are true:
    %
    \begin{enumerate}
        \item The map $e$ is differentiable in $t$, with
            %
            \begin{equation}\label{eq:matdiffl}
                \dot e(t) = A e(t) = e(t) A.
            \end{equation}
            %
        \item The fundamental theorem of calculus holds, in the sense that
            %
            \begin{equation}\label{eq:ftcexp}
                e(b) 
                 = e(a) + \int_a^b \dot e(s) \diff s \quad \text{for all } a \leq b.
            \end{equation}
            %
    \end{enumerate}
    %
\end{lemma}



For example, to prove (iv) we fix $A \in \lL(E)$ and consider $e(t) = \exp(t
A)$. Applying (ii) yields
%
\begin{equation*}
    \frac{e(t + h) - e(t)}{h}
    = \frac{\exp(tA + hA) - \exp(tA)}{h}
    = \exp(tA) \frac{\exp(hA) - I}{h}
\end{equation*}
%
Using the definition \eqref{eq:expfun} and taking the limit as $h \downarrow
0$ gives $\dot e(t) = e(t) A $.  A small variation on the argument above
shows that $\dot e(t) = A e(t)$ also holds.


\subsection{Initial Value Problems}

Now fix $A \in \lL(E)$ and consider the \navy{initial value problem} (IVP) $\dot
u = A u$ with $u_0 \in E$ fixed.  A solution to this problem is a continuous
function $s \mapsto u_s$ from $\RR_+$ to $E$ such that the IVP conditions hold.

\begin{proposition}\label{p:}
    Given $A \in \lL(E)$, the function $u_t = \exp(tA) u_0$ is the 
    unique solution to the IVP $\dot u = A u$ with $u_0 \in E$ fixed. 
\end{proposition}




\section{Introduction to Semigroups}

Add roadmap.

\subsection{Definitions}

Let $E$ be any set and let $(S_t) \coloneq (S_t)_{t \geq 0}$ be a family of
self-maps on $E$ with index $t \in \RR_+$. The pair $(E, (S_t))$ is called a \navy{semidynamical system} if 
$S_0$ is the idendity and $(S_t)$ has the semigroup property
%
\begin{equation*}
    S_{s + t} = S_t \circ S_s
    \quad \text{for all } s, t \in \RR_+.
\end{equation*}
%
Given $u \in E$, the function $t \mapsto S_t u$ is called the \navy{orbit} or
\navy{trajectory} of $u$ under $(S_t)$.

If $E$ is a vector space and each $S_t$ is linear, then $(E, (S_t))$ is called
an \navy{algebraic operator (AO) semigroup}.  If, in addition, $E$ is a Banach
space and $t \mapsto S_t u$ is continuous for all $u \in E$, then $(E, (S_t))$
is called a \navy{$C_0$-semigroup}.  When $E$ is understood, we say that $(S_t)$
is a $C_0$-semigroup.



\subsection{Continuity Results}

In what follows, $E$ is a real Banach space and $\lL(E)$ is the set of bounded linear
operators from $E$ to itself.  The symbol $\| \cdot \|$ denotes either the norm
on $E$ or the operator norm on $\lL(E)$, depending on context.

Let $K$ be a compact subset of $\RR$ and let $\{S_t\}_{t \in K}$ be a subset of
$\lL(E)$.  The following result is from \cite{engel2006short}.

\begin{lemma}\label{l:contin}
    The following statements are equivalent:
    %
    \begin{enumerate}
        \item The map $t \mapsto S_t u$ is continuous on $K$ for all $u \in E$.
        \item $\|S_t\|$ is bounded over $t \in K$ and there exists a dense subset
            $D$ of $E$ such that $t \mapsto S_t u$ is continuous on $K$ for all $u \in D$.
        \item For any compact $C \subset E$, the map $(t, u) \mapsto S_t u$ is
            uniformly continuous on $K \times C$.
    \end{enumerate}
\end{lemma}

\begin{proof}
    ((i) $\implies$ (ii)) By (i), for any $u \in E$, the map $t \mapsto S_t u$ is
    continuous on a compact set and, therefore, its image is bounded in $E$.  
    Hence, by the uniform boundedness principle, $\|S_t\|$ is bounded over $t
    \in K$.  The statement in (ii) regarding continuity is obvious.

    ((ii) $\implies$ (iii)).  Fix compact $C \subset E$ and $\epsilon > 0$.
    We metrize $K \times C$ by setting $d((s, u), (t, v)) = \| u - v\| \vee |s-t|$.
    Choose $M \in \NN$ such that $\|S_t\| \leq M$ for all $t \in K$.  Let $D$ be the
    dense set in (ii) and observe that the set of open balls $B(u, \epsilon/M)$
    over $u \in D$ provides an open cover of $C$.  As such, we can choose a finite
    set $D_F \subset D$ such that $C$ is contained in $\cup_{u \in D_F} B(u,
    \epsilon/M)$. Since, for each $u \in D_F$, the map
    $t \mapsto S_t u$ is continuous on a compact set, it is also uniformly
    continuous.  As a result, given $u \in D_F$, we can select a $\delta_u > 0$ such that
    %
    \begin{equation*}
        |s - t| < \delta_u \implies \| S_s u - S_t u \| < \epsilon.
    \end{equation*}
    %
    Let $\delta$ be the minimum of $\{\delta_u\}_{u \in D_F}$ and $\epsilon / M$.
    If we take $u, v \in C$ and $s, t \in K$ with $d((s, u), (t, v)) < \delta$, then,
    choosing $w \in D_F$ with $\|u - w \| < \epsilon / M$, we have
    %
    \begin{align*}
        \| S_s u - S_t v \|
        & \leq \| S_s u - S_s w \| 
            + \| S_s w - S_t w \| 
                + \| S_t w - S_t v \|
                \\
        & = \| S_s (u - w) \| 
            + \| S_s w - S_t w \| 
                + \| S_t (w - v) \|
                \\
        & < M (\epsilon / M) + \epsilon + M (2 \epsilon / M) = 4 \epsilon.
    \end{align*}
    %
    Hence $(t, u) \mapsto S_t u$ is
            uniformly continuous on $K \times C$, as claimed.

    ((iii) $\implies$ (i)) This implication is trivial (take $C$ to be a
    singleton).
\end{proof}

\begin{lemma}\label{l:ubfc}
    If $(S_t)_{t \geq 0}$ is a $C_0$-semigroup on $E$, then
        $\sup_{t \leq \delta} \| S_t \| < \infty$ for all $\delta > 0$.
\end{lemma}

\begin{proof}
    We first claim there exists an $\epsilon > 0$ such that $
    \sup_{t \leq \epsilon} \| S_t \| < \infty$.  Indeed, if no such $\epsilon$
    exists, then there exists a sequence $t_n \to 0$ such that 
    $\|S_{t_n}\|$ is unbounded.  But then, by the principle of uniform
    boundedness, there exists a $u \in E$ such that $\|S_{t_n} u\|$ is
    unbounded.  This contradicts the continuity property of $C_0$-semigroups.

    Now let $\epsilon$ be as above and choose $M \in \NN$ with $\| S_t \| \leq
    M$ whenever $t \leq \epsilon$.  Fix $k \in \NN$ and $t \leq k \epsilon$.
    Since $S_t$ is $k$ compositions of $S_{t/k}$, and since $t/k < \epsilon$, 
    the semigroup property yields $\| S_t \| \leq k M$.  Hence $t \mapsto S_t$
    is bounded on $[0, k \epsilon]$.  Since $k$ was an arbitrary element of
    $\NN$, this proves the claim in Lemma~\ref{l:ubfc}.
\end{proof}

\begin{lemma}\label{l:semiequivcon}
    An AO semigroup $(S_t)_{t \geq 0}$ on $E$ is a 
    $C_0$-semigroup on $E$ if and only if $\lim_{t \downarrow 0} S_t u = u$ for all $u \in E$.
\end{lemma}

\begin{proof}
    Sufficiency is obvious.  Regarding necessity, fix $u \in E$ and $t > 0$.  We
    need to show that $\|S_{t+h} u - S_t u\| \to 0$ as $h \to 0$.  Suppose first
    that $h \downarrow 0$.  Then
    %
    \begin{equation*}
        \|S_{t+h} u - S_t u\|  
        = \|S_t S_h u - S_t u\|  
        \leq \|S_t \| \| S_h u - u\|  \to 0.
    \end{equation*}
    %
    If, on the other hand $h \uparrow 0$, then
    %
    \begin{equation*}
        \|S_{t+h} u - S_t u\|  
        = \|S_{t+h} u - S_{t + h} S_{-h} u\|  
        \leq \|S_{t+h} \| \| u - S_{-h} u\| \to 0 .
    \end{equation*}
    %
    In the last step we used the fact that $\|S_{t+h} \| $ is bounded over $h$
    by Lemma~\ref{l:ubfc}.
\end{proof}

\begin{lemma}\label{l:aose}
    Let $(S_t)_{t \geq 0}$ be an AO semigroup on $E$.
    If there exists a dense subset $D$ of $E$ such that $\lim_{t \downarrow 0}
    S_t u = u$ for all $u \in D$ and, in addition, $\sup_{t \leq \delta} \|S_t
    \| < \infty$ for some $\delta > 0$, then $(S_t)_{t \geq 0}$ is a
    $C_0$-semigroup.
\end{lemma}

\begin{proof}
    Fix $u \in E$.  By Lemma~\ref{l:semiequivcon} it suffices to show
    that, for a given sequence $t_n \downarrow 0$, we have $S_{t_n} u \to u$ as
    $n \to 0$. 
    To see that this holds, fix $t_n \downarrow 0$ and choose a compact subset
    $K$ of $\RR_+$ such
    that $\{t_n\} \subset K$.  Since $K$ is compact, $K \ni t \mapsto S_t w$ is
    continuous when $w \in D$, and $\|S_t \|$ is bounded over $t \in K$,
    Lemma~\ref{l:contin} implies that $K \ni t \mapsto S_t u$ is continuous.
    In particular, $S_{t_n} u \to u$ as $n \to 0$. 
\end{proof}



\subsection{Examples}


\subsubsection{Left-Shift Semigroups}

Let $C_0(\RR_+)$ be the set of all continuous real-valued functions $f$ on
$\RR_+$ with $f(x) \to 0$ as $x \to \infty$.  The set $C_0(\RR_+)$ is paired
with the supremum norm.  Consider the \navy{left
translation semigroup} given by $(S_t f)(x) = f(x + t)$.

\begin{lemma}\label{l:ltsemi}
    $(S_t)$ is a $C_0$-semigroup on $C_0(\RR_+)$.  
\end{lemma}

\begin{proof}
    Evidently $S_0 f = f$.  The semigroup property holds because, for $s, t
    \geq 0$, we have
    %
    \begin{equation*}
        (S_{s + t} f)(x) = f(x + s + t) = (S_t (S_s f))(x).
    \end{equation*}
    %
    Regarding continuity, fix $f \in C_0(\RR_+)$ and let $(t_n)$ be a real sequence
    with $t_n \downarrow 0$.  Fix $\epsilon > 0$.  Since $f$ is uniformly
    continuous, we can select a $\delta > 0$ such that $|f(x) - f(y)| < \epsilon$
    whenever $|x-y|<\delta$.  Let $N \in \NN$ be such that $t_n < \delta$ when
    $n \geq N$.  Then, for $n \geq N$,
    %
    \begin{equation*}
        \| S_{t_n} f - f \|
        = \sup_x | f(x + t_n)  - f(x) |
        < \epsilon.
    \end{equation*}
    %
    Hence $S_t f \downarrow f$ and $(S_t)$ is a $C_0$-semigroup.
\end{proof}


Let $C_0^1(\RR_+)$ be the set of all continuously differentiable $f \in
C_0(\RR+)$ with $f' \in C_0(\RR+)$.  The set $C_0^1(\RR_+)$ is paired
with the norm $\|f\| = \sup_x |f(x)| + \sup_x |f'(x)|$.  

\begin{lemma}\label{l:ltsemi2}
    $(S_t)$ is a $C_0$-semigroup on $C_0^1(\RR_+)$.  
\end{lemma}

\begin{proof}
    In view of Lemma~\ref{l:ltsemi}, we only need to check continuity.
    Fixing $f \in C_0^1(\RR_+)$, we have
    %
    \begin{equation*}
        \| S_t f - f \|
        = \sup_x | f(x + t)  - f(x) | + \sup_x | f'(x + t)  - f'(x) |
    \end{equation*}
    %
    Since $f$ and $f'$ are both in $C_0(\RR_+)$, the proof of
    Lemma~\ref{l:ltsemi} implies that both terms on the right hand side
    converge to zero as $t \downarrow 0$.  Hence continuity holds.
\end{proof}

\subsubsection{Right-Shift Semigroups}\label{ss:rssemi}

Here we discuss right-shift semigroups.  We will embed them in a space of
integrable functions.  Below $\lambda$ denotes Lebesgue measure.

Let $C_c(\RR)$ be the set of all continuous real-valued functions $f$ on
$\RR$ that vanish off a compact set.  Let $L_1(\RR)$ be the set of 
Borel measurable real-valued functions on $\RR$ with $\| f \| \coloneq \int
|f| \diff \lambda < \infty$.  Let $S_t$ be the linear operator on $L_1(\RR)$
defined by 
%
\begin{equation*}
    (S_t f)(x) = f(x + t)
    \qquad (x \in \RR, \; t \geq 0).
\end{equation*}
%

\begin{lemma}\label{l:rtsemi}
    $(S_t)$ is a $C_0$-semigroup on $L_1(\RR)$.  
\end{lemma}

\begin{proof}
    It is simple to confirm that $(S_t)$ is an AO semigroup on $L_1(\RR)$.
    Regarding continuity, note that $\|S_t f\| = \int |f(x-t)| \diff x = \| f\|$, so
    $\|S_t\|$ is bounded in $t$.  Since $C_c(\RR)$ is dense in $L_1(\RR)$ under this
    norm, Lemma~\ref{l:aose} implies that, to show $(S_t)$ is a $C_0$-semigroup
    on $L_1(\RR)$, it suffices to show that $\| S_t f - f \| \to 0$ for any $f \in
    C_c(\RR)$.

    To this end, fix $f \in C_c(\RR)$ and let $K$ be a compact set such that $f$
    vanishes off $K$.  Fix $\epsilon > 0$.  By uniform continuity, we can take a
    $\delta > 0$ such that $|x-y| < \delta$ implies $|f(x) - f(y)| < \epsilon /
    \lambda(K)$.  If $t < \delta$, then
    %
    \begin{equation*}
        \| S_t f - f \| 
        = \int | f(x-t) - f(x) | \diff x
        \leq \lambda(K) \frac{\epsilon}{\lambda(K)} = \epsilon.
    \end{equation*}
    %
    This completes the proof of $C_0$-continuity of $(S_t)$ on $L_1(\RR)$.
\end{proof}


\subsubsection{Multiplication Semigroups}\label{ss:multisemi}

Let $(\Xsf, \bB, \mu)$ be a $\sigma$-finite measure space and let
$\phi$ be a measurable map from $\Xsf$ to $\RR_+$.  Define
%
\begin{equation*}
    S_t f = \exp(-t \phi(x)) f(x) 
    \qquad (x \in \Xsf, \; t \geq 0).
\end{equation*}
%
The family $(S_t)$ is called a \navy{multiplication semigroup}.

\begin{lemma}\label{l:multisemi}
    $(S_t)$ is a $C_0$-semigroup on $L_1(\Xsf, \bB, \mu)$.
\end{lemma}

\begin{proof}
    It is simple to confirm that $(S_t)$ is an AO semigroup on $L_1(\RR)$.
    Regarding continuity, fix $f \in L_1(\Xsf, \bB, \mu)$ and observe that
    %
    \begin{equation*}
        \| S_t f - f \|
        = \int |f(x)| | \exp(-t\phi(x)) - 1 | \mu(\diff x).
    \end{equation*}
    %
    It follows from the dominated convergence theorem that this integral
    converges to zero as $t \downarrow 0$.  Hence $(S_t)$ is a $C_0$-semigroup
    on $L_1(\Xsf, \bB, \mu)$.
\end{proof}



\subsubsection{Uniformly Continuous Semigroups}\label{ss:ucsemi}

Fixing $A \in \lL(E)$, consider the family of linear operators on $E$ given
by 
%
\begin{equation*}
    S_t u = \exp(t A) u
    \qquad (u \in E, \; t \geq 0)
\end{equation*}
%
We recall that the exponential function $\phi(t) \coloneq \exp(t A)$ 
%
\begin{enumerate}
    \item obeys $\phi(0) = I$ and $\phi(s + t) = \phi(t) \phi(s)$ for all $s, t
        \in \RR$; and
    \item is continuous as a map from $\RR$ to $\lL(E)$.
\end{enumerate}
%
From (i) we can easily confirm that $(S_t)$ is an algebraic operator semigroup
on $E$.  Regarding continuity, (i) and (ii) imply that
%
\begin{equation}\label{eq:ucsemi}
    \lim_{t \downarrow 0} \| S_t - I \| = 0.
\end{equation}
%
It follows from \eqref{eq:ucsemi} that $(S_t)$ is a $C_0$-semigroup on $E$.

Any operator semigroup $(S_t)$ on $E$ obeying \eqref{eq:ucsemi} is called a
\navy{uniformly continuous semigroup}.  In fact no other examples exist:

\begin{theorem}\label{t:ucsemi}
    If $(S_t)$ is a uniformly continuous semigroup on $E$, then there exists an
    $A \in \lL(E)$ such that $S_t u = \exp(t A) u$ for all $u \in E$ and $t \geq 0$.
\end{theorem}

The proof of Theorem~\ref{t:ucsemi} can be found 2(b) of \cite{engel2006short}.


\section{Generators and Resolvents}

Add roadmap.


\subsection{Infinitesmial Generators}

Let $E$ be a Banach space and let $(S_t)$ be an AO semigroup.  Set
%
\begin{equation*}
    \dD(A) =
    \left\{
        u \in E \st 
        \lim_{t \downarrow 0} \frac{S_t u - u}{t} \text{ exists}
    \right\}
    \quad \text{and} \quad
       Au = \lim_{t \downarrow 0} \frac{S_t u - u}{t} \text{ on }
       \dD(A).
\end{equation*}

The map $A$ is called the \navy{infinitesimal generator} of $(S_t)$ and $\dD(A)$
is its domain.  

\begin{lemma}\label{l:dals}
    The set $\dD(A)$ is a linear subspace of $E$ and $A$ is linear on $\dD(A)$.
\end{lemma}

\begin{proof}
    Fix $\alpha, \beta \in \RR$ and $u,v \in \dD(A)$.  Let $w = \alpha u + \beta v$.
    Since
    %
    \begin{equation*}
        \frac{S_t w - w}{t} 
        = \alpha \frac{S_t  u - u}{t} + \beta \frac{S_t  v - v}{t},
    \end{equation*}
    %
    we see that $w \in \dD(A)$ and, moreover, $Aw = \alpha Au + \beta Av$.
\end{proof}

The next result shows that any trajectory of $(S_t)$ starting from a point $\bar
u$ in $\dD(A)$ obeys a linear differential equation evolving in $\dD(A)$, with $A$
as the ``vector field.''


\begin{theorem}\label{t:diffpath}
    If $(S_t)$ is a $C_0$-semigroup, then $\dD(A)$ is dense in $E$.  Moreover,
    if $u_0 \in E$ and $u_t = S_t u_0$ for all $t \geq 0$, then, for all $t \geq
    0$,
    %
    \begin{equation}\label{eq:pathfrome}
        \int_0^t u_s \diff s \in \dD(A) \quad \text{and} \quad
        u_t = u_0 + A \int_0^t u_s \diff s.
    \end{equation}
    %
\end{theorem}


\begin{proof}
    We begin by proving \eqref{eq:pathfrome}. Fix $u_0 \in E$ and let $u_t$ be
    as defined above.  For $h > 0$,
    %
    \begin{align*}
            S_h \int_0^t u_s \diff s - \int_0^t u_s \diff s
        & = 
            S_h \int_0^t S_s u_0 \diff s - \int_0^t S_s u_0 \diff s
        \\
        & = 
            \int_0^t S_{h + s} u_0 \diff s - \int_0^t S_s u_0 \diff s
        \\
        & = 
            \int_h^{t+h} u_s \diff s - \int_0^t u_s \diff s
        = 
            \int_t^{t+h} u_s \diff s - \int_0^h u_s \diff s
    \end{align*}
    %
    Applying Lemma~\ref{l:riemann}, we see that multiplying by $(1/h)$ and taking the limit as $h
    \downarrow 0$ gives $A \int_0^t u_s \diff s = u_t - u_0$. Thus
    \eqref{eq:pathfrome} is verified.

    To see that $\dD(A)$ is dense, let $v_h = (1/h) \int_0^h u_s \diff s$ for
    all $h > 0$. By Lemma~\ref{l:riemann},  $v_h \to u_0$ as $h \downarrow 0$.
    Moreover, by the preceding results in this proof, $v_h$ is in $\dD(A)$ for
    all $h > 0$.   Hence $\dD(A)$ is dense in $E$.
\end{proof}


\begin{theorem}\label{t:diffpath}
    If $(S_t)$ is a $C_0$-semigroup, $u_0 \in \dD(A)$ and $u_t = S_t u_0$
    for all $t \geq 0$, then
    %
    \begin{enumerate}
        \item $u_t \in \dD(A)$  for all $t \geq 0$,
        \item the derivative $\dot u_t$ exists and equals $A u_t$, and
    \end{enumerate}
    %
    In particular, the trajectory $t \mapsto u_t$ is a solution to the initial
    value problem $\dot u_t = A u_t$ with initial condition $u_0 \in \dD(A)$.  Moreover,
    this trajectory obeys
    %
    \begin{equation}
        u_t = u_0 + \int_0^t S_s A u_0 \diff s
        \quad \text{for all } t \geq 0.
    \end{equation}
    %
\end{theorem}

\begin{proof}
    Fix $u_0 \in \dD(A)$ and let $u_t = S_t u_0$. Claim (i) follows from (ii)
    because, when $t$ is fixed and $s \mapsto u_s$ is differentiable, the limit
    %
    \begin{equation*}
        \lim_{h \downarrow 0}
         \frac{S_h u_t - u_t}{h}
         =
        \lim_{h \downarrow 0}
         \frac{u_{t+h} - u_t}{h} 
    \end{equation*}
    %
    exists (and hence $u_t \in \dD(A)$).
    Regarding (ii), fix $t > 0$  (the case $t=0$ is trivial) and $h > 0$.  Using
    continuity of $S_t$ and taking $h \downarrow 0$, we have
    %
    \begin{equation*}
         \frac{u_{t+h} - u_t}{h} = S_t \frac{u_h - u_0}{h} \to S_t A u_0.
    \end{equation*}
    %
    Regarding the left-hand derivative, we take $0 < h < t$ and use
    %
    \begin{equation*}
        \frac{u_{t-h} - u_t}{-h} - S_t A u_0
        = S_{t-h} \left( \frac{u_0 - u_h}{-h} - S_h A u_0 \right)
    \end{equation*}
    %
    and the boundedness of $\| S_t \|$ over bounded sets (Lemma~\ref{l:ubfc}) to
    obtain a finite $M$ such that
    %
    \begin{equation*}
        \left\|
            \frac{u_{t-h} - u_t}{-h} - S_t A u_0
        \right\|
        \leq M
        \left\|
            \frac{u_0 - u_h}{-h} - A u_0
        \right\|
        +
        M
        \left\|
            A u_0 - S_h A u_0
        \right\|.
    \end{equation*}
    %
    Both terms on the right converge to zero in $h$, so we have proved that
    the derivative $\dot u_t$ of $u_t$ exists and is equal to $S_t A u_0$.

    Since $u_t = S_t u_0$, our proof will be complete if we can show that 
    %
    \begin{equation}\label{eq:stuat}
        A S_t u_0 = S_t A u_0 
        \qquad (t \geq 0, \; u_0 \in \dD(A))
    \end{equation}
    %
    But this follows easily from 
    %
    \begin{equation*}
        A S_t u_0
        = \lim_{h \downarrow 0} \frac{S_h S_t u_0 - S_t u_0}{h}
        = \lim_{h \downarrow 0} \frac{S_t S_h u_0 - S_t u_0}{h}
        = \lim_{h \downarrow 0} S_t \frac{S_h u_0 - u_0}{h}
    \end{equation*}
    %
    and continuity of $t \mapsto S_t$.
\end{proof}


\bibliographystyle{apalike}
\bibliography{sgs_bib}


\end{document}

