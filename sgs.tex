\documentclass[12pt, reqno]{amsart}


\usepackage{amsmath, amssymb, amsthm, amsfonts}

% avoid `Too many math alphabets used in version normal` error
\newcommand\hmmax{0}
\newcommand\bmmax{0}

\usepackage[mathscr]{euscript} 
\usepackage{stmaryrd} % St Mary's Road symbols font --- some extra symbols

%\usepackage{fontspec} 
\usepackage[xcharter]{newtxmath}
%\setmainfont{XCharter}

\usepackage{graphics, stackrel}
\usepackage{graphicx}

\usepackage{verbatim}
\usepackage{natbib}
\usepackage{enumitem}

%font
%\usepackage{lmodern}
%\usepackage[T1]{fontenc}
%\usepackage{fontspec} 
%\usepackage[xcharter]{newtxmath}
%\setmainfont{XCharter}
%\usepackage{mathpazo}
%\usepackage{tgpagella}

%subfloats / figures
\usepackage{caption}
\usepackage{subcaption}

% For pandas latex tables
\usepackage{booktabs}


\usepackage{fancyvrb}
\usepackage[dvipsnames,svgnames,table]{xcolor}
\usepackage{mdwlist}

\usepackage[breaklinks=true, citecolor=Brown, colorlinks=true, linkcolor=blue]{hyperref}

% lists
\usepackage{enumitem}
\setlist[enumerate]{itemsep=2pt,topsep=3pt}
\setlist[itemize]{itemsep=2pt,topsep=3pt}
\setlist[enumerate,1]{label=(\roman*)}

\usepackage{mathrsfs}  % caligraphic
%\usepackage{stix} 
\usepackage{bbm}
\usepackage{bm}        % bold symbols


%% page layout
\usepackage[left=1.25in, right=1.25in, top=1.0in, bottom=1.15in, includehead, includefoot]{geometry}

% nice inequalities
\renewcommand{\leq}{\leqslant}
\renewcommand{\geq}{\geqslant}

% inner product
\providecommand{\inner}[1]{\left\langle{#1}\right\rangle}
\providecommand{\innerp}[1]{\left\langle{#1}\right\rangle_\pi}

% lists
\usepackage{enumitem}
\setlist[enumerate]{itemsep=2pt,topsep=3pt}
\setlist[itemize]{itemsep=2pt,topsep=3pt}
\setlist[enumerate,1]{label={\upshape (\roman*)}}


\usepackage[ruled, linesnumbered]{algorithm2e}

%extra spacing
\renewcommand{\baselinestretch}{1.25}

%horizonal line
\newcommand{\HRule}{\rule{\linewidth}{0.3mm}}

% skip a line between paragraphs, no indentation
%\setlength{\parskip}{1.5ex plus0.5ex minus0.5ex}
%\setlength{\parindent}{0pt}

% footnote without a maker (blfootnote)
\newcommand\blfootnote[1]{%
  \begingroup
  \renewcommand\thefootnote{}\footnote{#1}%
  \addtocounter{footnote}{-1}%
  \endgroup
}

\DeclareMathOperator{\fix}{fix}
\DeclareMathOperator{\Span}{span}
\DeclareMathOperator{\diag}{diag}
\DeclareMathOperator*{\argmin}{arg\,min}
\DeclareMathOperator*{\argmax}{arg\,max}

\DeclareMathOperator{\cl}{cl}
\DeclareMathOperator{\Int}{int}
%\DeclareMathOperator{\overset{\circ}}{int}
\DeclareMathOperator{\Prob}{Prob}
\DeclareMathOperator{\determinant}{det}
\DeclareMathOperator{\Var}{Var}
\DeclareMathOperator{\Cov}{Cov}
\DeclareMathOperator{\graph}{graph}

% mics short cuts and symbols
\newcommand{\st}{\ensuremath{\ \mathrm{s.t.}\ }}
\newcommand{\setntn}[2]{ \{ #1 : #2 \} }
\newcommand{\fore}{\therefore \quad}
\newcommand{\preqsd}{\preceq_{sd} }
\newcommand{\toas}{\stackrel {\textrm{ \scriptsize{a.s.} }} {\to} }
\newcommand{\tod}{\stackrel { d } {\to} }
\newcommand{\too}{\stackrel { o } {\to} }
\newcommand{\toweak}{\stackrel { w } {\to} }
\newcommand{\topr}{\stackrel { p } {\to} }
\newcommand{\disteq}{\stackrel { \mathscr D } {=} }
\newcommand{\eqdist}{\stackrel {\textrm{ \scriptsize{d} }} {=} }
\newcommand{\iidsim}{\stackrel {\textrm{ {\sc iid }}} {\sim} }
\newcommand{\1}{\mathbbm 1}
\newcommand{\la}{\langle}
\newcommand{\ra}{\rangle}
\newcommand{\dee}{\,{\rm d}}
\newcommand{\og}{{\mathbbm G}}
\newcommand{\ctimes}{\! \times \!}
\newcommand{\sint}{{\textstyle\int}}

\newcommand{\given}{\, | \,}
\newcommand{\A}{\forall}

% d for integrals
\newcommand*\diff{\mathop{}\!\mathrm{d}}
\newcommand*\e{\mathrm{e}}

% nice emptyset
\let\oldemptyset\emptyset
\let\emptyset\varnothing


%\renewcommand{\times}{\! \times \!}

\newcommand{\aA}{\mathcal A}
\newcommand{\cC}{\mathscr C}
\newcommand{\sS}{\mathcal S}
\newcommand{\bB}{\mathcal B}
\newcommand{\oO}{\mathcal O}
\newcommand{\gG}{\mathcal G}
\newcommand{\hH}{\mathcal H}
\newcommand{\kK}{\mathcal K}
\newcommand{\iI}{\mathcal I}
\newcommand{\eE}{\mathcal E}
\newcommand{\fF}{\mathscr F}
\newcommand{\qQ}{\mathcal Q}
\newcommand{\tT}{\mathcal T}
\newcommand{\xX}{\mathcal X}
\newcommand{\yY}{\mathcal Y}
\newcommand{\rR}{\mathcal R}
\newcommand{\zZ}{\mathcal Z}
\newcommand{\wW}{\mathcal W}
\newcommand{\uU}{\mathcal U}
\newcommand{\lL}{\mathcal L}
\newcommand{\mM}{\mathcal M}
\newcommand{\dD}{\mathcal D}
\newcommand{\pP}{\mathcal P}
\newcommand{\vV}{\mathcal V}

\newcommand{\Bsf}{\mathsf B}
\newcommand{\Hsf}{\mathsf H}
\newcommand{\Vsf}{\mathsf V}

\newcommand{\BB}{\mathbbm B}
\newcommand{\DD}{\mathbbm D}
\newcommand{\RR}{\mathbbm R}
\newcommand{\CC}{\mathbbm C}
\newcommand{\QQ}{\mathbbm Q}
\newcommand{\NN}{\mathbbm N}
\newcommand{\GG}{\mathbbm G}
\newcommand{\UU}{\mathbbm U}
\newcommand{\TT}{\mathbbm T}
\newcommand{\YY}{\mathbbm Y}
\newcommand{\ZZ}{\mathbbm Z}
\newcommand{\HH}{\mathbbm H}
\newcommand{\MM}{\mathbbm M}
\newcommand{\PP}{\mathbbm P}
\newcommand{\VV}{\mathbbm V}
\newcommand{\EE}{\mathbbm E}


\newcommand{\bH}{\mathbf H}
\newcommand{\bT}{\mathbf T}

\newcommand{\var}{\mathbbm V}

\newcommand{\Xsf}{\mathsf X}
\newcommand{\Msf}{\mathsf M}
\newcommand{\Asf}{\mathsf A}
\newcommand{\Gsf}{\mathsf G}
\newcommand{\II}{\mathsf I}
\newcommand{\WW}{\mathsf W}

\renewcommand{\phi}{\varphi}
\renewcommand{\epsilon}{\varepsilon}

\newcommand{\bP}{\mathbf P}
\newcommand{\bQ}{\mathbf Q}
\newcommand{\bE}{\mathbf E}
\newcommand{\bM}{\mathbf M}
\newcommand{\bX}{\mathbf X}
\newcommand{\bY}{\mathbf Y}

\theoremstyle{plain}
\newtheorem{theorem}{Theorem}[section]
\newtheorem{corollary}[theorem]{Corollary}
\newtheorem{lemma}[theorem]{Lemma}
\newtheorem{proposition}[theorem]{Proposition}


\theoremstyle{definition}
\newtheorem{definition}{Definition}[section]
\newtheorem{axiom}{Axiom}[section]
\newtheorem{example}{Example}[section]
\newtheorem{remark}{Remark}[section]
\newtheorem{notation}{Notation}[section]
\newtheorem{assumption}{Assumption}[section]
\newtheorem{condition}{Condition}[section]


%\DeclareTextFontCommand{\emph}{\bfseries}

%%%%%%%%%%%%%%%%%% end my preamble %%%%%%%%%%%%%%%%%%%%%%%%%%%%%%%%%%%


\newcommand{\navy}[1]{\textcolor{blue}{#1}}




\begin{document}


\title{Semigroup Notes}

\author{JS and JY}

\date{\today}

\maketitle

\section{Definitions}

Let $E$ be any set and let $\TT \coloneq (T_t)_{t \geq 0}$ be a family of
self-maps on $E$. The pair $(E, \TT)$ is called a \navy{semidynamical system} if 
$T_0$ is the idendity and $\TT$ has the semigroup property
%
\begin{equation*}
    T_{s + t} = T_t \circ T_s
    \quad \text{for all } s, t \in \RR_+.
\end{equation*}
%
If $E$ is a vector space and each $T_t \in \TT$ is linear, then $(E, \TT)$ is
called an \navy{algebraic operator (AO) semigroup}.  If, in addition, $E$ is an
Banach space and $t \mapsto T_t u$ is continuous for all $u \in E$, then $(E, \TT)$
is called a \navy{$C_0$-semigroup}.  When $E$ is understood, we say that $\TT$
is a $C_0$-semigroup.



\section{Continuity results}

In what follows, $E$ is a Banach space and $\lL(E)$ is the set of bounded linear
operators from $E$ to itself.  The symbol $\| \cdot \|$ denotes either the norm
on $E$ or the operator norm on $\lL(E)$, depending on context.

Let $K$ be a compact subset of $\RR$ and let $\{T_t\}_{t \in K}$ be a subset of
$\lL(E)$.  The following result is from \cite{engel2006short}.

\begin{lemma}\label{l:contin}
    The following statements are equivalent:
    %
    \begin{enumerate}
        \item The map $t \mapsto T_t u$ is continuous on $K$ for all $u \in E$.
        \item $\|T_t\|$ is bounded over $t \in K$ and there exists a dense subset
            $D$ of $E$ such that $t \mapsto T_t u$ is continuous on $K$ for all $u \in D$.
        \item For any compact $C \subset E$, the map $(t, u) \mapsto T_t u$ is
            uniformly continuous on $K \times C$.
    \end{enumerate}
\end{lemma}

\begin{proof}
    ((i) $\implies$ (ii)) By (i), for any $u \in E$, the map $t \mapsto T_t u$ is
    continuous on a compact set and, therefore, its image is bounded in $E$.  
    Hence, by the uniform boundedness principle, $\|T_t\|$ is bounded over $t
    \in K$.  The statement in (ii) regarding continuity is obvious.

    ((ii) $\implies$ (iii)).  Fix compact $C \subset E$ and $\epsilon > 0$.
    We metrize $K \times C$ by setting $d((s, u), (t, v)) = \| u - v\| \vee |s-t|$.
    Choose $M$ such that $\|T_t\| \leq M$ for all $t \in K$.  Let $D$ be the
    dense set in (ii) and observe that the set of open balls $B(u, \epsilon/M)$
    over $u \in D$ provides an open cover of $C$.  As such, we can choose a finite
    set $D_F \subset D$ such that $C$ is contained in $\cup_{u \in D_F} B(u,
    \epsilon/M)$. Since, for each $u \in D_F$, the map
    $t \mapsto T_t u$ is continuous on a compact set, it is also uniformly
    continous.  As a result, we can select a $\delta_u > 0$ such that
    %
    \begin{equation*}
        |s - t| < \delta_u \implies \| T_s u - T_t u \| < \epsilon.
    \end{equation*}
    %
    Let $\delta$ be the minimum of $\{\delta_u\}_{u \in D_F}$ and $\epsilon / M$.
    If we take $u, v \in C$ and $s, t \in K$ with $d((s, u), (t, v)) < \delta$, then,
    choosing $w \in D_F$ with $\|u - w \| < \epsilon / M$, we have
    %
    \begin{align*}
        \| T_s u - T_t v \|
        & \leq \| T_s u - T_s w \| 
            + \| T_s w - T_t w \| 
                + \| T_t w - T_t v \|
                \\
        & = \| T_s (u - w) \| 
            + \| T_s w - T_t w \| 
                + \| T_t (w - v) \|
        < M (\epsilon / M) + \epsilon + M (2 \epsilon / M) = 4 \epsilon.
    \end{align*}
    %
    Hence $(t, u) \mapsto T_t u$ is
            uniformly continuous on $K \times C$, as claimed.

    ((iii) $\implies$ (i)) This claim is also obvious (take $C$ to be a
    singleton).
\end{proof}

\begin{lemma}\label{l:ubfc}
    If $(T_t)_{t \geq 0}$ is a $C_0$-semigroup on $E$, then
        $\sup_{t \leq \delta} \| T_t \| < \infty$ for all $\delta > 0$.
\end{lemma}

\begin{proof}
    We first claim there exists an $\epsilon > 0$ such that $
    \sup_{t \leq \epsilon} \| T_t \| < \infty$.  Indeed, if no such $\epsilon$
    exists, then there exists a sequence $t_n \to 0$ such that 
    $\|T_{t_n}\|$ is unbounded.  But then, by the principle of uniform
    boundedness, there exists a $u \in E$ such that $\|T_{t_n} u\|$ is
    unbounded.  This contradicts the continuity property of $C_0$-semigroups.

    Now let $\epsilon$ be as above and choose $M \in \NN$ with $\| T_t \| \leq
    M$ whenever $t \leq \epsilon$.  Fix $k \in \NN$ and $t \leq k \epsilon$.
    Since $T_t$ is $k$ compositions of $T_{t/k}$, and since $t/k < \epsilon$, 
    the semigroup property yields $\| T_t \| \leq k M$.  Hence $t \mapsto T_t$
    is bounded on $[0, k \epsilon]$.  Since $k$ was an arbitrary element of
    $\NN$, this proves the claim in Lemma~\ref{l:ubfc}.
\end{proof}

\begin{lemma}\label{l:semiequivcon}
    An AO semigroup $(T_t)_{t \geq 0}$ on $E$ is a 
    $C_0$-semigroup on $E$ if and only if $\lim_{t \downarrow 0} T_t u = u$ for all $u \in E$.
\end{lemma}

\begin{proof}
    Sufficiency is obvious.  Regarding necessity, fix $u \in E$ and $t > 0$.  We
    need to show that $\|T_{t+h} u - T_t u\| \to 0$ as $h \to 0$.  Suppose first
    that $h \downarrow 0$.  Then
    %
    \begin{equation*}
        \|T_{t+h} u - T_t u\|  
        = \|T_t T_h u - T_t u\|  
        \leq \|T_t \| \| T_h u - u\|  \to 0.
    \end{equation*}
    %
    If, on the other hand $h \uparrow 0$, then
    %
    \begin{equation*}
        \|T_{t+h} u - T_t u\|  
        = \|T_{t+h} u - T_{t + h} T_{-h} u\|  
        \leq \|T_{t+h} \| \| u - T_{-h} u\| \to 0 .
    \end{equation*}
    %
    In the last step we used the fact that $\|T_{t+h} \| $ is bounded over $h$
    by Lemma~\ref{l:ubfc}.
\end{proof}

\begin{lemma}\label{l:aose}
    Let $(T_t)_{t \geq 0}$ be an AO semigroup on $E$.
    If there exists a dense subset $D$ of $E$ such that $\lim_{t \downarrow 0}
    T_t u = u$ for all $u \in D$ and, in addition, $\sup_{t \leq \delta} \|T_t
    \| < \infty$ for some $\delta > 0$, then $(T_t)_{t \geq 0}$ is a
    $C_0$-semigroup on $E$.
\end{lemma}

\begin{proof}
    Fix $u \in E$.  In view of Lemma~\ref{l:semiequivcon} it suffices to show
    that, for a given sequence $t_n \downarrow 0$, we have $T_{t_n} u \to u$ as
    $n \to 0$. 

    To see that this holds, fix $t_n \downarrow 0$ and choose a compact set $K$ such
    that $\{t_n\} \subset K$.  Since $K$ is compact, $K \ni t \mapsto T_t w$ is
    continuous when $w \in D$, and $\|T_t \|$ is bounded over $t \in K$,
    Lemma~\ref{l:contin} implies that $K \ni t \mapsto T_t u$ is continuous.
    In particular, $T_{t_n} u \to u$ as $n \to 0$. 
\end{proof}

\bibliographystyle{apalike}
\bibliography{sgs_bib}


\end{document}

